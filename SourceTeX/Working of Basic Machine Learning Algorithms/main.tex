\documentclass[11pt, a4paper]{book}

    \usepackage[breakable]{tcolorbox}
    \usepackage{parskip} % Stop auto-indenting (to mimic markdown behaviour)
    \usepackage[margin=1in]{geometry}
    \usepackage{iftex}
    
    \ifPDFTeX
    	\usepackage[T1]{fontenc}
    	\usepackage{mathpazo}
    \else
    	\usepackage{fontspec}
    \fi

    % Basic figure setup, for now with no caption control since it's done
    % automatically by Pandoc (which extracts ![](path) syntax from Markdown).
    \usepackage{graphicx}
    % Maintain compatibility with old templates. Remove in nbconvert 6.0
    \let\Oldincludegraphics\includegraphics
    % Ensure that by default, figures have no caption (until we provide a
    % proper Figure object with a Caption API and a way to capture that
    % in the conversion process - todo).
    \usepackage{caption}
    \DeclareCaptionFormat{nocaption}{}
    \captionsetup{format=nocaption,aboveskip=0pt,belowskip=0pt}

    \usepackage[Export]{adjustbox} % Used to constrain images to a maximum size
    \adjustboxset{max size={0.9\linewidth}{0.9\paperheight}}
    \usepackage{float}
    \floatplacement{figure}{H} % forces figures to be placed at the correct location
    \usepackage{xcolor} % Allow colors to be defined
    \usepackage{enumerate} % Needed for markdown enumerations to work
    \usepackage{geometry} % Used to adjust the document margins
    \usepackage{amsmath} % Equations
    \usepackage{amssymb} % Equations
    \usepackage{textcomp} % defines textquotesingle
    % Hack from http://tex.stackexchange.com/a/47451/13684:
    \AtBeginDocument{%
        \def\PYZsq{\textquotesingle}% Upright quotes in Pygmentized code
    }
    \usepackage{upquote} % Upright quotes for verbatim code
    \usepackage{eurosym} % defines \euro
    \usepackage[mathletters]{ucs} % Extended unicode (utf-8) support
    \usepackage{fancyvrb} % verbatim replacement that allows latex
    \usepackage{grffile} % extends the file name processing of package graphics 
                         % to support a larger range
    \makeatletter % fix for grffile with XeLaTeX
    \def\Gread@@xetex#1{%
      \IfFileExists{"\Gin@base".bb}%
      {\Gread@eps{\Gin@base.bb}}%
      {\Gread@@xetex@aux#1}%
    }
    \makeatother

    % The hyperref package gives us a pdf with properly built
    % internal navigation ('pdf bookmarks' for the table of contents,
    % internal cross-reference links, web links for URLs, etc.)
    \usepackage{hyperref}
    % The default LaTeX title has an obnoxious amount of whitespace. By default,
    % titling removes some of it. It also provides customization options.
    \usepackage{titling}
    \usepackage{longtable} % longtable support required by pandoc >1.10
    \usepackage{booktabs}  % table support for pandoc > 1.12.2
    \usepackage[inline]{enumitem} % IRkernel/repr support (it uses the enumerate* environment)
    \usepackage[normalem]{ulem} % ulem is needed to support strikethroughs (\sout)
                                % normalem makes italics be italics, not underlines
    \usepackage{mathrsfs}
    

    
    % Colors for the hyperref package
    \definecolor{urlcolor}{rgb}{0,.145,.698}
    \definecolor{linkcolor}{rgb}{.71,0.21,0.01}
    \definecolor{citecolor}{rgb}{.12,.54,.11}

    % ANSI colors
    \definecolor{ansi-black}{HTML}{3E424D}
    \definecolor{ansi-black-intense}{HTML}{282C36}
    \definecolor{ansi-red}{HTML}{E75C58}
    \definecolor{ansi-red-intense}{HTML}{B22B31}
    \definecolor{ansi-green}{HTML}{00A250}
    \definecolor{ansi-green-intense}{HTML}{007427}
    \definecolor{ansi-yellow}{HTML}{DDB62B}
    \definecolor{ansi-yellow-intense}{HTML}{B27D12}
    \definecolor{ansi-blue}{HTML}{208FFB}
    \definecolor{ansi-blue-intense}{HTML}{0065CA}
    \definecolor{ansi-magenta}{HTML}{D160C4}
    \definecolor{ansi-magenta-intense}{HTML}{A03196}
    \definecolor{ansi-cyan}{HTML}{60C6C8}
    \definecolor{ansi-cyan-intense}{HTML}{258F8F}
    \definecolor{ansi-white}{HTML}{C5C1B4}
    \definecolor{ansi-white-intense}{HTML}{A1A6B2}
    \definecolor{ansi-default-inverse-fg}{HTML}{FFFFFF}
    \definecolor{ansi-default-inverse-bg}{HTML}{000000}

    % commands and environments needed by pandoc snippets
    % extracted from the output of `pandoc -s`
    \providecommand{\tightlist}{%
      \setlength{\itemsep}{0pt}\setlength{\parskip}{0pt}}
    \DefineVerbatimEnvironment{Highlighting}{Verbatim}{commandchars=\\\{\}}
    % Add ',fontsize=\small' for more characters per line
    \newenvironment{Shaded}{}{}
    \newcommand{\KeywordTok}[1]{\textcolor[rgb]{0.00,0.44,0.13}{\textbf{{#1}}}}
    \newcommand{\DataTypeTok}[1]{\textcolor[rgb]{0.56,0.13,0.00}{{#1}}}
    \newcommand{\DecValTok}[1]{\textcolor[rgb]{0.25,0.63,0.44}{{#1}}}
    \newcommand{\BaseNTok}[1]{\textcolor[rgb]{0.25,0.63,0.44}{{#1}}}
    \newcommand{\FloatTok}[1]{\textcolor[rgb]{0.25,0.63,0.44}{{#1}}}
    \newcommand{\CharTok}[1]{\textcolor[rgb]{0.25,0.44,0.63}{{#1}}}
    \newcommand{\StringTok}[1]{\textcolor[rgb]{0.25,0.44,0.63}{{#1}}}
    \newcommand{\CommentTok}[1]{\textcolor[rgb]{0.38,0.63,0.69}{\textit{{#1}}}}
    \newcommand{\OtherTok}[1]{\textcolor[rgb]{0.00,0.44,0.13}{{#1}}}
    \newcommand{\AlertTok}[1]{\textcolor[rgb]{1.00,0.00,0.00}{\textbf{{#1}}}}
    \newcommand{\FunctionTok}[1]{\textcolor[rgb]{0.02,0.16,0.49}{{#1}}}
    \newcommand{\RegionMarkerTok}[1]{{#1}}
    \newcommand{\ErrorTok}[1]{\textcolor[rgb]{1.00,0.00,0.00}{\textbf{{#1}}}}
    \newcommand{\NormalTok}[1]{{#1}}
    
    % Additional commands for more recent versions of Pandoc
    \newcommand{\ConstantTok}[1]{\textcolor[rgb]{0.53,0.00,0.00}{{#1}}}
    \newcommand{\SpecialCharTok}[1]{\textcolor[rgb]{0.25,0.44,0.63}{{#1}}}
    \newcommand{\VerbatimStringTok}[1]{\textcolor[rgb]{0.25,0.44,0.63}{{#1}}}
    \newcommand{\SpecialStringTok}[1]{\textcolor[rgb]{0.73,0.40,0.53}{{#1}}}
    \newcommand{\ImportTok}[1]{{#1}}
    \newcommand{\DocumentationTok}[1]{\textcolor[rgb]{0.73,0.13,0.13}{\textit{{#1}}}}
    \newcommand{\AnnotationTok}[1]{\textcolor[rgb]{0.38,0.63,0.69}{\textbf{\textit{{#1}}}}}
    \newcommand{\CommentVarTok}[1]{\textcolor[rgb]{0.38,0.63,0.69}{\textbf{\textit{{#1}}}}}
    \newcommand{\VariableTok}[1]{\textcolor[rgb]{0.10,0.09,0.49}{{#1}}}
    \newcommand{\ControlFlowTok}[1]{\textcolor[rgb]{0.00,0.44,0.13}{\textbf{{#1}}}}
    \newcommand{\OperatorTok}[1]{\textcolor[rgb]{0.40,0.40,0.40}{{#1}}}
    \newcommand{\BuiltInTok}[1]{{#1}}
    \newcommand{\ExtensionTok}[1]{{#1}}
    \newcommand{\PreprocessorTok}[1]{\textcolor[rgb]{0.74,0.48,0.00}{{#1}}}
    \newcommand{\AttributeTok}[1]{\textcolor[rgb]{0.49,0.56,0.16}{{#1}}}
    \newcommand{\InformationTok}[1]{\textcolor[rgb]{0.38,0.63,0.69}{\textbf{\textit{{#1}}}}}
    \newcommand{\WarningTok}[1]{\textcolor[rgb]{0.38,0.63,0.69}{\textbf{\textit{{#1}}}}}
    
    
    % Define a nice break command that doesn't care if a line doesn't already
    % exist.
    \def\br{\hspace*{\fill} \\* }
    % Math Jax compatibility definitions
    \def\gt{>}
    \def\lt{<}
    \let\Oldtex\TeX
    \let\Oldlatex\LaTeX
    \renewcommand{\TeX}{\textrm{\Oldtex}}
    \renewcommand{\LaTeX}{\textrm{\Oldlatex}}
    % Document parameters
    % Document title
    \title{Working of Basic Machine Learning Algorithms}
    
    \author{Richeek Das}
    
    
    
% Pygments definitions
\makeatletter
\def\PY@reset{\let\PY@it=\relax \let\PY@bf=\relax%
    \let\PY@ul=\relax \let\PY@tc=\relax%
    \let\PY@bc=\relax \let\PY@ff=\relax}
\def\PY@tok#1{\csname PY@tok@#1\endcsname}
\def\PY@toks#1+{\ifx\relax#1\empty\else%
    \PY@tok{#1}\expandafter\PY@toks\fi}
\def\PY@do#1{\PY@bc{\PY@tc{\PY@ul{%
    \PY@it{\PY@bf{\PY@ff{#1}}}}}}}
\def\PY#1#2{\PY@reset\PY@toks#1+\relax+\PY@do{#2}}

\expandafter\def\csname PY@tok@w\endcsname{\def\PY@tc##1{\textcolor[rgb]{0.73,0.73,0.73}{##1}}}
\expandafter\def\csname PY@tok@c\endcsname{\let\PY@it=\textit\def\PY@tc##1{\textcolor[rgb]{0.25,0.50,0.50}{##1}}}
\expandafter\def\csname PY@tok@cp\endcsname{\def\PY@tc##1{\textcolor[rgb]{0.74,0.48,0.00}{##1}}}
\expandafter\def\csname PY@tok@k\endcsname{\let\PY@bf=\textbf\def\PY@tc##1{\textcolor[rgb]{0.00,0.50,0.00}{##1}}}
\expandafter\def\csname PY@tok@kp\endcsname{\def\PY@tc##1{\textcolor[rgb]{0.00,0.50,0.00}{##1}}}
\expandafter\def\csname PY@tok@kt\endcsname{\def\PY@tc##1{\textcolor[rgb]{0.69,0.00,0.25}{##1}}}
\expandafter\def\csname PY@tok@o\endcsname{\def\PY@tc##1{\textcolor[rgb]{0.40,0.40,0.40}{##1}}}
\expandafter\def\csname PY@tok@ow\endcsname{\let\PY@bf=\textbf\def\PY@tc##1{\textcolor[rgb]{0.67,0.13,1.00}{##1}}}
\expandafter\def\csname PY@tok@nb\endcsname{\def\PY@tc##1{\textcolor[rgb]{0.00,0.50,0.00}{##1}}}
\expandafter\def\csname PY@tok@nf\endcsname{\def\PY@tc##1{\textcolor[rgb]{0.00,0.00,1.00}{##1}}}
\expandafter\def\csname PY@tok@nc\endcsname{\let\PY@bf=\textbf\def\PY@tc##1{\textcolor[rgb]{0.00,0.00,1.00}{##1}}}
\expandafter\def\csname PY@tok@nn\endcsname{\let\PY@bf=\textbf\def\PY@tc##1{\textcolor[rgb]{0.00,0.00,1.00}{##1}}}
\expandafter\def\csname PY@tok@ne\endcsname{\let\PY@bf=\textbf\def\PY@tc##1{\textcolor[rgb]{0.82,0.25,0.23}{##1}}}
\expandafter\def\csname PY@tok@nv\endcsname{\def\PY@tc##1{\textcolor[rgb]{0.10,0.09,0.49}{##1}}}
\expandafter\def\csname PY@tok@no\endcsname{\def\PY@tc##1{\textcolor[rgb]{0.53,0.00,0.00}{##1}}}
\expandafter\def\csname PY@tok@nl\endcsname{\def\PY@tc##1{\textcolor[rgb]{0.63,0.63,0.00}{##1}}}
\expandafter\def\csname PY@tok@ni\endcsname{\let\PY@bf=\textbf\def\PY@tc##1{\textcolor[rgb]{0.60,0.60,0.60}{##1}}}
\expandafter\def\csname PY@tok@na\endcsname{\def\PY@tc##1{\textcolor[rgb]{0.49,0.56,0.16}{##1}}}
\expandafter\def\csname PY@tok@nt\endcsname{\let\PY@bf=\textbf\def\PY@tc##1{\textcolor[rgb]{0.00,0.50,0.00}{##1}}}
\expandafter\def\csname PY@tok@nd\endcsname{\def\PY@tc##1{\textcolor[rgb]{0.67,0.13,1.00}{##1}}}
\expandafter\def\csname PY@tok@s\endcsname{\def\PY@tc##1{\textcolor[rgb]{0.73,0.13,0.13}{##1}}}
\expandafter\def\csname PY@tok@sd\endcsname{\let\PY@it=\textit\def\PY@tc##1{\textcolor[rgb]{0.73,0.13,0.13}{##1}}}
\expandafter\def\csname PY@tok@si\endcsname{\let\PY@bf=\textbf\def\PY@tc##1{\textcolor[rgb]{0.73,0.40,0.53}{##1}}}
\expandafter\def\csname PY@tok@se\endcsname{\let\PY@bf=\textbf\def\PY@tc##1{\textcolor[rgb]{0.73,0.40,0.13}{##1}}}
\expandafter\def\csname PY@tok@sr\endcsname{\def\PY@tc##1{\textcolor[rgb]{0.73,0.40,0.53}{##1}}}
\expandafter\def\csname PY@tok@ss\endcsname{\def\PY@tc##1{\textcolor[rgb]{0.10,0.09,0.49}{##1}}}
\expandafter\def\csname PY@tok@sx\endcsname{\def\PY@tc##1{\textcolor[rgb]{0.00,0.50,0.00}{##1}}}
\expandafter\def\csname PY@tok@m\endcsname{\def\PY@tc##1{\textcolor[rgb]{0.40,0.40,0.40}{##1}}}
\expandafter\def\csname PY@tok@gh\endcsname{\let\PY@bf=\textbf\def\PY@tc##1{\textcolor[rgb]{0.00,0.00,0.50}{##1}}}
\expandafter\def\csname PY@tok@gu\endcsname{\let\PY@bf=\textbf\def\PY@tc##1{\textcolor[rgb]{0.50,0.00,0.50}{##1}}}
\expandafter\def\csname PY@tok@gd\endcsname{\def\PY@tc##1{\textcolor[rgb]{0.63,0.00,0.00}{##1}}}
\expandafter\def\csname PY@tok@gi\endcsname{\def\PY@tc##1{\textcolor[rgb]{0.00,0.63,0.00}{##1}}}
\expandafter\def\csname PY@tok@gr\endcsname{\def\PY@tc##1{\textcolor[rgb]{1.00,0.00,0.00}{##1}}}
\expandafter\def\csname PY@tok@ge\endcsname{\let\PY@it=\textit}
\expandafter\def\csname PY@tok@gs\endcsname{\let\PY@bf=\textbf}
\expandafter\def\csname PY@tok@gp\endcsname{\let\PY@bf=\textbf\def\PY@tc##1{\textcolor[rgb]{0.00,0.00,0.50}{##1}}}
\expandafter\def\csname PY@tok@go\endcsname{\def\PY@tc##1{\textcolor[rgb]{0.53,0.53,0.53}{##1}}}
\expandafter\def\csname PY@tok@gt\endcsname{\def\PY@tc##1{\textcolor[rgb]{0.00,0.27,0.87}{##1}}}
\expandafter\def\csname PY@tok@err\endcsname{\def\PY@bc##1{\setlength{\fboxsep}{0pt}\fcolorbox[rgb]{1.00,0.00,0.00}{1,1,1}{\strut ##1}}}
\expandafter\def\csname PY@tok@kc\endcsname{\let\PY@bf=\textbf\def\PY@tc##1{\textcolor[rgb]{0.00,0.50,0.00}{##1}}}
\expandafter\def\csname PY@tok@kd\endcsname{\let\PY@bf=\textbf\def\PY@tc##1{\textcolor[rgb]{0.00,0.50,0.00}{##1}}}
\expandafter\def\csname PY@tok@kn\endcsname{\let\PY@bf=\textbf\def\PY@tc##1{\textcolor[rgb]{0.00,0.50,0.00}{##1}}}
\expandafter\def\csname PY@tok@kr\endcsname{\let\PY@bf=\textbf\def\PY@tc##1{\textcolor[rgb]{0.00,0.50,0.00}{##1}}}
\expandafter\def\csname PY@tok@bp\endcsname{\def\PY@tc##1{\textcolor[rgb]{0.00,0.50,0.00}{##1}}}
\expandafter\def\csname PY@tok@fm\endcsname{\def\PY@tc##1{\textcolor[rgb]{0.00,0.00,1.00}{##1}}}
\expandafter\def\csname PY@tok@vc\endcsname{\def\PY@tc##1{\textcolor[rgb]{0.10,0.09,0.49}{##1}}}
\expandafter\def\csname PY@tok@vg\endcsname{\def\PY@tc##1{\textcolor[rgb]{0.10,0.09,0.49}{##1}}}
\expandafter\def\csname PY@tok@vi\endcsname{\def\PY@tc##1{\textcolor[rgb]{0.10,0.09,0.49}{##1}}}
\expandafter\def\csname PY@tok@vm\endcsname{\def\PY@tc##1{\textcolor[rgb]{0.10,0.09,0.49}{##1}}}
\expandafter\def\csname PY@tok@sa\endcsname{\def\PY@tc##1{\textcolor[rgb]{0.73,0.13,0.13}{##1}}}
\expandafter\def\csname PY@tok@sb\endcsname{\def\PY@tc##1{\textcolor[rgb]{0.73,0.13,0.13}{##1}}}
\expandafter\def\csname PY@tok@sc\endcsname{\def\PY@tc##1{\textcolor[rgb]{0.73,0.13,0.13}{##1}}}
\expandafter\def\csname PY@tok@dl\endcsname{\def\PY@tc##1{\textcolor[rgb]{0.73,0.13,0.13}{##1}}}
\expandafter\def\csname PY@tok@s2\endcsname{\def\PY@tc##1{\textcolor[rgb]{0.73,0.13,0.13}{##1}}}
\expandafter\def\csname PY@tok@sh\endcsname{\def\PY@tc##1{\textcolor[rgb]{0.73,0.13,0.13}{##1}}}
\expandafter\def\csname PY@tok@s1\endcsname{\def\PY@tc##1{\textcolor[rgb]{0.73,0.13,0.13}{##1}}}
\expandafter\def\csname PY@tok@mb\endcsname{\def\PY@tc##1{\textcolor[rgb]{0.40,0.40,0.40}{##1}}}
\expandafter\def\csname PY@tok@mf\endcsname{\def\PY@tc##1{\textcolor[rgb]{0.40,0.40,0.40}{##1}}}
\expandafter\def\csname PY@tok@mh\endcsname{\def\PY@tc##1{\textcolor[rgb]{0.40,0.40,0.40}{##1}}}
\expandafter\def\csname PY@tok@mi\endcsname{\def\PY@tc##1{\textcolor[rgb]{0.40,0.40,0.40}{##1}}}
\expandafter\def\csname PY@tok@il\endcsname{\def\PY@tc##1{\textcolor[rgb]{0.40,0.40,0.40}{##1}}}
\expandafter\def\csname PY@tok@mo\endcsname{\def\PY@tc##1{\textcolor[rgb]{0.40,0.40,0.40}{##1}}}
\expandafter\def\csname PY@tok@ch\endcsname{\let\PY@it=\textit\def\PY@tc##1{\textcolor[rgb]{0.25,0.50,0.50}{##1}}}
\expandafter\def\csname PY@tok@cm\endcsname{\let\PY@it=\textit\def\PY@tc##1{\textcolor[rgb]{0.25,0.50,0.50}{##1}}}
\expandafter\def\csname PY@tok@cpf\endcsname{\let\PY@it=\textit\def\PY@tc##1{\textcolor[rgb]{0.25,0.50,0.50}{##1}}}
\expandafter\def\csname PY@tok@c1\endcsname{\let\PY@it=\textit\def\PY@tc##1{\textcolor[rgb]{0.25,0.50,0.50}{##1}}}
\expandafter\def\csname PY@tok@cs\endcsname{\let\PY@it=\textit\def\PY@tc##1{\textcolor[rgb]{0.25,0.50,0.50}{##1}}}

\def\PYZbs{\char`\\}
\def\PYZus{\char`\_}
\def\PYZob{\char`\{}
\def\PYZcb{\char`\}}
\def\PYZca{\char`\^}
\def\PYZam{\char`\&}
\def\PYZlt{\char`\<}
\def\PYZgt{\char`\>}
\def\PYZsh{\char`\#}
\def\PYZpc{\char`\%}
\def\PYZdl{\char`\$}
\def\PYZhy{\char`\-}
\def\PYZsq{\char`\'}
\def\PYZdq{\char`\"}
\def\PYZti{\char`\~}
% for compatibility with earlier versions
\def\PYZat{@}
\def\PYZlb{[}
\def\PYZrb{]}
\makeatother


    % For linebreaks inside Verbatim environment from package fancyvrb. 
    \makeatletter
        \newbox\Wrappedcontinuationbox 
        \newbox\Wrappedvisiblespacebox 
        \newcommand*\Wrappedvisiblespace {\textcolor{red}{\textvisiblespace}} 
        \newcommand*\Wrappedcontinuationsymbol {\textcolor{red}{\llap{\tiny$\m@th\hookrightarrow$}}} 
        \newcommand*\Wrappedcontinuationindent {3ex } 
        \newcommand*\Wrappedafterbreak {\kern\Wrappedcontinuationindent\copy\Wrappedcontinuationbox} 
        % Take advantage of the already applied Pygments mark-up to insert 
        % potential linebreaks for TeX processing. 
        %        {, <, #, %, $, ' and ": go to next line. 
        %        _, }, ^, &, >, - and ~: stay at end of broken line. 
        % Use of \textquotesingle for straight quote. 
        \newcommand*\Wrappedbreaksatspecials {% 
            \def\PYGZus{\discretionary{\char`\_}{\Wrappedafterbreak}{\char`\_}}% 
            \def\PYGZob{\discretionary{}{\Wrappedafterbreak\char`\{}{\char`\{}}% 
            \def\PYGZcb{\discretionary{\char`\}}{\Wrappedafterbreak}{\char`\}}}% 
            \def\PYGZca{\discretionary{\char`\^}{\Wrappedafterbreak}{\char`\^}}% 
            \def\PYGZam{\discretionary{\char`\&}{\Wrappedafterbreak}{\char`\&}}% 
            \def\PYGZlt{\discretionary{}{\Wrappedafterbreak\char`\<}{\char`\<}}% 
            \def\PYGZgt{\discretionary{\char`\>}{\Wrappedafterbreak}{\char`\>}}% 
            \def\PYGZsh{\discretionary{}{\Wrappedafterbreak\char`\#}{\char`\#}}% 
            \def\PYGZpc{\discretionary{}{\Wrappedafterbreak\char`\%}{\char`\%}}% 
            \def\PYGZdl{\discretionary{}{\Wrappedafterbreak\char`\$}{\char`\$}}% 
            \def\PYGZhy{\discretionary{\char`\-}{\Wrappedafterbreak}{\char`\-}}% 
            \def\PYGZsq{\discretionary{}{\Wrappedafterbreak\textquotesingle}{\textquotesingle}}% 
            \def\PYGZdq{\discretionary{}{\Wrappedafterbreak\char`\"}{\char`\"}}% 
            \def\PYGZti{\discretionary{\char`\~}{\Wrappedafterbreak}{\char`\~}}% 
        } 
        % Some characters . , ; ? ! / are not pygmentized. 
        % This macro makes them "active" and they will insert potential linebreaks 
        \newcommand*\Wrappedbreaksatpunct {% 
            \lccode`\~`\.\lowercase{\def~}{\discretionary{\hbox{\char`\.}}{\Wrappedafterbreak}{\hbox{\char`\.}}}% 
            \lccode`\~`\,\lowercase{\def~}{\discretionary{\hbox{\char`\,}}{\Wrappedafterbreak}{\hbox{\char`\,}}}% 
            \lccode`\~`\;\lowercase{\def~}{\discretionary{\hbox{\char`\;}}{\Wrappedafterbreak}{\hbox{\char`\;}}}% 
            \lccode`\~`\:\lowercase{\def~}{\discretionary{\hbox{\char`\:}}{\Wrappedafterbreak}{\hbox{\char`\:}}}% 
            \lccode`\~`\?\lowercase{\def~}{\discretionary{\hbox{\char`\?}}{\Wrappedafterbreak}{\hbox{\char`\?}}}% 
            \lccode`\~`\!\lowercase{\def~}{\discretionary{\hbox{\char`\!}}{\Wrappedafterbreak}{\hbox{\char`\!}}}% 
            \lccode`\~`\/\lowercase{\def~}{\discretionary{\hbox{\char`\/}}{\Wrappedafterbreak}{\hbox{\char`\/}}}% 
            \catcode`\.\active
            \catcode`\,\active 
            \catcode`\;\active
            \catcode`\:\active
            \catcode`\?\active
            \catcode`\!\active
            \catcode`\/\active 
            \lccode`\~`\~ 	
        }
    \makeatother

    \let\OriginalVerbatim=\Verbatim
    \makeatletter
    \renewcommand{\Verbatim}[1][1]{%
        %\parskip\z@skip
        \sbox\Wrappedcontinuationbox {\Wrappedcontinuationsymbol}%
        \sbox\Wrappedvisiblespacebox {\FV@SetupFont\Wrappedvisiblespace}%
        \def\FancyVerbFormatLine ##1{\hsize\linewidth
            \vtop{\raggedright\hyphenpenalty\z@\exhyphenpenalty\z@
                \doublehyphendemerits\z@\finalhyphendemerits\z@
                \strut ##1\strut}%
        }%
        % If the linebreak is at a space, the latter will be displayed as visible
        % space at end of first line, and a continuation symbol starts next line.
        % Stretch/shrink are however usually zero for typewriter font.
        \def\FV@Space {%
            \nobreak\hskip\z@ plus\fontdimen3\font minus\fontdimen4\font
            \discretionary{\copy\Wrappedvisiblespacebox}{\Wrappedafterbreak}
            {\kern\fontdimen2\font}%
        }%
        
        % Allow breaks at special characters using \PYG... macros.
        \Wrappedbreaksatspecials
        % Breaks at punctuation characters . , ; ? ! and / need catcode=\active 	
        \OriginalVerbatim[#1,codes*=\Wrappedbreaksatpunct]%
    }
    \makeatother

    % Exact colors from NB
    \definecolor{incolor}{HTML}{303F9F}
    \definecolor{outcolor}{HTML}{D84315}
    \definecolor{cellborder}{HTML}{CFCFCF}
    \definecolor{cellbackground}{HTML}{F7F7F7}
    
    % prompt
    \makeatletter
    \newcommand{\boxspacing}{\kern\kvtcb@left@rule\kern\kvtcb@boxsep}
    \makeatother
    \newcommand{\prompt}[4]{
        \ttfamily\llap{{\color{#2}[#3]:\hspace{3pt}#4}}\vspace{-\baselineskip}
    }
    

    
    % Prevent overflowing lines due to hard-to-break entities
    \sloppy 
    % Setup hyperref package
    \hypersetup{
      breaklinks=true,  % so long urls are correctly broken across lines
      colorlinks=true,
      urlcolor=urlcolor,
      linkcolor=linkcolor,
      citecolor=citecolor,
      }
    % Slightly bigger margins than the latex defaults
    
    \geometry{verbose,tmargin=1in,bmargin=1in,lmargin=1in,rmargin=1in}
    
    

\begin{document}
    \setlength{\abovedisplayskip}{3pt}
    \setlength{\belowdisplayskip}{3pt}
    \maketitle
    
\chapter{Working of Basic Machine Learning Algorithms}

    
    \(\large\textbf{Things we will look at here :}\)

\begin{itemize}
\item
  \textbf{Linear Regression}
\item
  \textbf{Gradient Descent}
\item
  \textbf{Polynomial Regression}
\item
  \textbf{Learning Curves}
\item
  \textbf{Regularized Linear Models}
\item
  \textbf{Logistic Regression}
\item
  \textbf{Softmax Regression}
\item
  \textbf{Decision Tree}
\item
  \textbf{Random Forest}
\end{itemize}

    \hypertarget{linear-regression}{%
\section{Linear Regression}\label{linear-regression}}

A linear model makes a prediction by simply computing a weighted
\textbf{sum} of the \textbf{input features}, plus a constant called the
bias term (also called the \textbf{intercept term})

\textbf{\emph{Linear Regression}} model prediction :

\[\hat{y} = \theta_{0} + \theta_{1}x_{1} + \theta_{2}x_{2} + \dotsm + \theta_{n}x_{n}\]

Here :

\begin{itemize}
\item
  \(\hat{y}\) is the predicted value
\item
  \(n\) is the number of features
\item
  \(x_{i}\) is the \(i^{th}\) feature value
\item
  \(\theta_{j}\) is the \(j^{th}\) model parameter, i.e the feature
  weights.
\end{itemize}

It can be further elegantly written as :

\[\hat{y} = h_{\theta}(\textbf{x}) = \theta^{\intercal}\cdot\textbf{x}\]

Here :

\begin{itemize}
\item
  \(\theta\) is the model's \emph{parameter column vector} containing
  the bias term \(\theta_{0}\) and the feature weights.
\item
  \(\textbf{x}\) is the instance's \emph{feature vector} containing
  \(x_{0}\) to \(x_{n}\), with \(x_{0}\) always equal to 1
\item
  Needless to say the \(\theta\cdot\textbf{x}\) is the usual dot product
\item
  \(h_{\theta}\) is the hypothesis function, using the model parameters
  \(\theta\)
\end{itemize}

\textbf{So, let's come to the algorithms handling the training part of
Linear Regression}

The most common performance measure of a regression model is the
\textbf{Root Mean Square Error (RMSE)}. Therefore, to train a
\textbf{Linear Regression} model, we need to find the value of
\(\theta\) that minimizes the \textbf{RMSE}. It is simpler to minimize
the \textbf{mean squared error (MSE)} than the \textbf{RMSE}, and it
leads to the same result.

The \textbf{MSE} of a \textbf{LR} hypothesis \(h_{\theta}\) on a
training set \(\mathbf{X}\) is calculated using :

\[\mathbf{MSE(X},h_{\theta}) = \frac{1}{m}\sum_{i=1}^{m} \Bigg( \theta^{\intercal}\mathbf{x}^{(i)} - y^{(i)} \Bigg)^2\]

This function mentioned above is called the \textbf{cost function} for a
\textbf{Linear Regression model}

To find the value of \(\theta\) that minimizes the cost function, there
is a mathematical tool that gives the result directly. Its called the
\emph{Normal Equation}

\hypertarget{the-normal-equation}{%
\subsection{The Normal Equation}\label{the-normal-equation}}

\[\hat{\theta} = \Big( \mathbf{X^{\intercal}X} \Big)^{-1} \mathbf{X^{\intercal}}\textbf{y}\]

Here :

\begin{itemize}
\item
  \(\hat{\theta}\) is the value of \(\theta\) that minimizes the cost
  function.
\item
  \(\mathbf{y}\) is the vector of target values containing \(y^{(1)}\)
  to \(y^{(m)}\).
\end{itemize}

Let's test this normal equation with some code !

    \begin{tcolorbox}[breakable, size=fbox, boxrule=1pt, pad at break*=1mm,colback=cellbackground, colframe=cellborder]
\prompt{In}{incolor}{1}{\boxspacing}
\begin{Verbatim}[commandchars=\\\{\}]
\PY{k+kn}{import} \PY{n+nn}{numpy} \PY{k}{as} \PY{n+nn}{np}

\PY{n}{X} \PY{o}{=} \PY{l+m+mi}{2} \PY{o}{*} \PY{n}{np}\PY{o}{.}\PY{n}{random}\PY{o}{.}\PY{n}{rand}\PY{p}{(}\PY{l+m+mi}{100}\PY{p}{,}\PY{l+m+mi}{1}\PY{p}{)}
\PY{n}{y} \PY{o}{=} \PY{l+m+mi}{5} \PY{o}{+} \PY{l+m+mi}{7} \PY{o}{*} \PY{n}{X} \PY{o}{+} \PY{n}{np}\PY{o}{.}\PY{n}{random}\PY{o}{.}\PY{n}{randn}\PY{p}{(}\PY{l+m+mi}{100}\PY{p}{,}\PY{l+m+mi}{1}\PY{p}{)}
\PY{c+c1}{\PYZsh{}\PYZsh{} Generates a random linear dataset with some Gaussian Noise}
\end{Verbatim}
\end{tcolorbox}

    Let's plot it :

    \begin{tcolorbox}[breakable, size=fbox, boxrule=1pt, pad at break*=1mm,colback=cellbackground, colframe=cellborder]
\prompt{In}{incolor}{2}{\boxspacing}
\begin{Verbatim}[commandchars=\\\{\}]
\PY{k+kn}{import} \PY{n+nn}{matplotlib}\PY{n+nn}{.}\PY{n+nn}{pyplot} \PY{k}{as} \PY{n+nn}{plt}

\PY{n}{plt}\PY{o}{.}\PY{n}{plot}\PY{p}{(}\PY{n}{X}\PY{p}{,} \PY{n}{y}\PY{p}{,} \PY{l+s+s2}{\PYZdq{}}\PY{l+s+s2}{ bo}\PY{l+s+s2}{\PYZdq{}}\PY{p}{,} \PY{n}{markersize} \PY{o}{=} \PY{l+m+mi}{3}\PY{p}{)}
\PY{n}{plt}\PY{o}{.}\PY{n}{xlim}\PY{p}{(}\PY{l+m+mf}{0.00}\PY{p}{,} \PY{l+m+mf}{2.00}\PY{p}{)}
\PY{n}{plt}\PY{o}{.}\PY{n}{grid}\PY{p}{(}\PY{k+kc}{True}\PY{p}{)}
\PY{n}{plt}\PY{o}{.}\PY{n}{xlabel}\PY{p}{(}\PY{l+s+s2}{\PYZdq{}}\PY{l+s+s2}{X}\PY{l+s+s2}{\PYZdq{}}\PY{p}{)}
\PY{n}{plt}\PY{o}{.}\PY{n}{ylabel}\PY{p}{(}\PY{l+s+s2}{\PYZdq{}}\PY{l+s+s2}{y}\PY{l+s+s2}{\PYZdq{}}\PY{p}{)}
\PY{n}{plt}\PY{o}{.}\PY{n}{title}\PY{p}{(}\PY{l+s+s2}{\PYZdq{}}\PY{l+s+s2}{Random Linear Dataset}\PY{l+s+s2}{\PYZdq{}}\PY{p}{)}
\PY{n}{plt}\PY{o}{.}\PY{n}{show}\PY{p}{(}\PY{p}{)}
\end{Verbatim}
\end{tcolorbox}

    \begin{center}
    \adjustimage{max size={0.9\linewidth}{0.9\paperheight}}{output_4_0.png}
    \end{center}
    { \hspace*{\fill} \\}
    
    We will use the \texttt{inv()} function from \emph{NumPy}'s linear
algebra module (\texttt{np.linalg}) to compute the inverse of a matrix,
and the \texttt{dot()} method for matrix multiplication :

    \begin{tcolorbox}[breakable, size=fbox, boxrule=1pt, pad at break*=1mm,colback=cellbackground, colframe=cellborder]
\prompt{In}{incolor}{3}{\boxspacing}
\begin{Verbatim}[commandchars=\\\{\}]
\PY{n}{X\PYZus{}b} \PY{o}{=} \PY{n}{np}\PY{o}{.}\PY{n}{c\PYZus{}}\PY{p}{[}\PY{n}{np}\PY{o}{.}\PY{n}{ones}\PY{p}{(}\PY{p}{(}\PY{l+m+mi}{100}\PY{p}{,} \PY{l+m+mi}{1}\PY{p}{)}\PY{p}{)}\PY{p}{,} \PY{n}{X}\PY{p}{]} \PY{c+c1}{\PYZsh{} add x0 = 1 to each instance}
\PY{n}{theta\PYZus{}best} \PY{o}{=} \PY{n}{np}\PY{o}{.}\PY{n}{linalg}\PY{o}{.}\PY{n}{inv}\PY{p}{(}\PY{n}{X\PYZus{}b}\PY{o}{.}\PY{n}{T}\PY{o}{.}\PY{n}{dot}\PY{p}{(}\PY{n}{X\PYZus{}b}\PY{p}{)}\PY{p}{)}\PY{o}{.}\PY{n}{dot}\PY{p}{(}\PY{n}{X\PYZus{}b}\PY{o}{.}\PY{n}{T}\PY{p}{)}\PY{o}{.}\PY{n}{dot}\PY{p}{(}\PY{n}{y}\PY{p}{)}
\end{Verbatim}
\end{tcolorbox}

    Function intended for : \(y = 5 + 7x\)

Expression found :

    \begin{tcolorbox}[breakable, size=fbox, boxrule=1pt, pad at break*=1mm,colback=cellbackground, colframe=cellborder]
\prompt{In}{incolor}{4}{\boxspacing}
\begin{Verbatim}[commandchars=\\\{\}]
\PY{n+nb}{print}\PY{p}{(}\PY{n}{theta\PYZus{}best}\PY{p}{[}\PY{p}{[}\PY{l+m+mi}{0}\PY{p}{]}\PY{p}{]}\PY{p}{)} \PY{c+c1}{\PYZsh{}\PYZsh{} Intercept term}
\PY{n+nb}{print}\PY{p}{(}\PY{n}{theta\PYZus{}best}\PY{p}{[}\PY{p}{[}\PY{l+m+mi}{1}\PY{p}{]}\PY{p}{]}\PY{p}{)} \PY{c+c1}{\PYZsh{}\PYZsh{} First weight term}
\end{Verbatim}
\end{tcolorbox}

    \begin{Verbatim}[commandchars=\\\{\}]
[[4.80833324]]
[[7.12658528]]
    \end{Verbatim}

    Now, let's get the predicted linear function using the \(\hat{\theta}\)
found using normal equations :

    \begin{tcolorbox}[breakable, size=fbox, boxrule=1pt, pad at break*=1mm,colback=cellbackground, colframe=cellborder]
\prompt{In}{incolor}{5}{\boxspacing}
\begin{Verbatim}[commandchars=\\\{\}]
\PY{n}{X\PYZus{}new} \PY{o}{=} \PY{n}{np}\PY{o}{.}\PY{n}{array}\PY{p}{(}\PY{p}{[}\PY{p}{[}\PY{l+m+mi}{0}\PY{p}{]}\PY{p}{,} \PY{p}{[}\PY{l+m+mi}{2}\PY{p}{]}\PY{p}{]}\PY{p}{)}
\PY{n}{X\PYZus{}new\PYZus{}b} \PY{o}{=} \PY{n}{np}\PY{o}{.}\PY{n}{c\PYZus{}}\PY{p}{[}\PY{n}{np}\PY{o}{.}\PY{n}{ones}\PY{p}{(}\PY{p}{(}\PY{l+m+mi}{2}\PY{p}{,} \PY{l+m+mi}{1}\PY{p}{)}\PY{p}{)}\PY{p}{,} \PY{n}{X\PYZus{}new}\PY{p}{]}
\PY{n}{y\PYZus{}predict} \PY{o}{=} \PY{n}{X\PYZus{}new\PYZus{}b}\PY{o}{.}\PY{n}{dot}\PY{p}{(}\PY{n}{theta\PYZus{}best}\PY{p}{)}
\end{Verbatim}
\end{tcolorbox}

    Let's plot the prediction :

    \begin{tcolorbox}[breakable, size=fbox, boxrule=1pt, pad at break*=1mm,colback=cellbackground, colframe=cellborder]
\prompt{In}{incolor}{6}{\boxspacing}
\begin{Verbatim}[commandchars=\\\{\}]
\PY{n}{plt}\PY{o}{.}\PY{n}{plot}\PY{p}{(}\PY{n}{X}\PY{p}{,} \PY{n}{y}\PY{p}{,} \PY{l+s+s2}{\PYZdq{}}\PY{l+s+s2}{ bo}\PY{l+s+s2}{\PYZdq{}}\PY{p}{,} \PY{n}{markersize} \PY{o}{=} \PY{l+m+mi}{3}\PY{p}{,} \PY{n}{label} \PY{o}{=} \PY{l+s+s2}{\PYZdq{}}\PY{l+s+s2}{Scattered Dataset}\PY{l+s+s2}{\PYZdq{}}\PY{p}{)}
\PY{n}{plt}\PY{o}{.}\PY{n}{xlim}\PY{p}{(}\PY{l+m+mf}{0.00}\PY{p}{,} \PY{l+m+mf}{2.00}\PY{p}{)}
\PY{n}{plt}\PY{o}{.}\PY{n}{grid}\PY{p}{(}\PY{k+kc}{True}\PY{p}{)}
\PY{n}{plt}\PY{o}{.}\PY{n}{xlabel}\PY{p}{(}\PY{l+s+s2}{\PYZdq{}}\PY{l+s+s2}{X}\PY{l+s+s2}{\PYZdq{}}\PY{p}{)}
\PY{n}{plt}\PY{o}{.}\PY{n}{ylabel}\PY{p}{(}\PY{l+s+s2}{\PYZdq{}}\PY{l+s+s2}{y}\PY{l+s+s2}{\PYZdq{}}\PY{p}{)}
\PY{n}{plt}\PY{o}{.}\PY{n}{plot}\PY{p}{(}\PY{n}{X\PYZus{}new}\PY{p}{,} \PY{n}{y\PYZus{}predict}\PY{p}{,} \PY{l+s+s2}{\PYZdq{}}\PY{l+s+s2}{r\PYZhy{}}\PY{l+s+s2}{\PYZdq{}}\PY{p}{,} \PY{n}{label} \PY{o}{=} \PY{l+s+s2}{\PYZdq{}}\PY{l+s+s2}{Predicted}\PY{l+s+s2}{\PYZdq{}}\PY{p}{)}
\PY{n}{plt}\PY{o}{.}\PY{n}{legend}\PY{p}{(}\PY{n}{loc} \PY{o}{=} \PY{l+s+s2}{\PYZdq{}}\PY{l+s+s2}{lower right}\PY{l+s+s2}{\PYZdq{}}\PY{p}{)}
\PY{n}{plt}\PY{o}{.}\PY{n}{show}\PY{p}{(}\PY{p}{)}
\end{Verbatim}
\end{tcolorbox}

    \begin{center}
    \adjustimage{max size={0.9\linewidth}{0.9\paperheight}}{output_12_0.png}
    \end{center}
    { \hspace*{\fill} \\}
    
    In \emph{NumPy} we usually calculate \(\hat{\theta}\) using :
\[\hat{\theta} = \mathbf{X}^{+}y\]

where, \(\mathbf{X}^{+}\) is the \textbf{pseudoinverse} of
\(\mathbf{X}\), aka the \textbf{Moore-Penrose inverse}. The
pseudoinverse is calculated using a standard matrix factorization
technique called \emph{Singular Value Decomposition(SVD)} \textgreater{}
Can be found in the mathematical preliminary \href{www.github.com}{part}

This approach is \textbf{more efficient} than computing the \emph{Normal
Equation}, plus it also handles the edge cases. The \emph{Normal
Equation} may not work if the matrix \(\mathbf{X^{\intercal}X}\) is not
invertible, such as if its not a square matrix or if some features are
redundant, but the pseudoinverse is always defined.

\textbf{REMEMBER TO PUT THE LINK HERE}

    \hypertarget{computational-complexity}{%
\subsection{Computational
Complexity}\label{computational-complexity}}

The \textbf{Normal Equation} computes the inverse of
\(\mathbf{X^{\intercal}X}\), which is an \((n + 1) \times (n + 1)\) matrix
(where \(n\) is the number of features). The computational complexity of
inverting such a matrix is typically about \(O(n^3)\), depending on the
implementation.

The \textbf{SVD} approach has \(O(n^2)\). Both these approaches are
linear with regard to the number of instances in the training set (they
are \(O(m)\)), so they handle large training sets efficiently.

    \hypertarget{gradient-descent}{%
\section{Gradient Descent}\label{gradient-descent}}

\emph{Gradient Descent} is a generic optimization algorithm capable of
finding optimal solutions to a wide range of problems.

    \begin{center}
    \adjustimage{max size={0.7\linewidth}{0.9\paperheight}}{download1.png}
    \end{center}
    { \hspace*{\fill} \\}

Here, we start by filling \(\theta\) with random values (\emph{random
initialization}). Then we improve it gradually, taking one \(d\theta\)
step at a time, each step making an attempt to reduce the cost function
till it converges to a minimum.

\emph{Learning Rate hyperparameter} : It speaks about the size of the
steps taken. if learning rate is too small then the algorithm will make
a huge amount of iterations to converge(in-efficient). On the other
hand, if its too high, then it might even make the algorithm diverge,
and fail to reach a good solution.

Well, one problem that arises at once, is the fact that there can be
multiple local minimas, thus making it difficult to find the global
minima with \emph{random initialization} plan, we previously had.

Fortunately, the \textbf{MSE cost function} for a \textbf{Linear
Regression} model happens to be a convex function (we can easily verify
it) ! This implies that there are no local minima, just one global
minimum. It is also a continuous function with a slope that never
changes abruptly. These two facts have a great consequence:
\textbf{Gradient Descent} is guaranteed to approach arbitrarily close
the global minimum(well, but excluding other factors).

When using Gradient Descent, we should ensure that all features have a
similar scale (e.g in \textbf{Python}, using
\texttt{Scikit-Learn\ StandardScaler} class)(there are other scalers as
well, which suit to different occasions), or else it will take much
longer to converge.

    \hypertarget{batch-gradient-descent}{%
\subsection{Batch Gradient Descent}\label{batch-gradient-descent}}

To implement \emph{GD} we nee dto compute the gradient of the \emph{cost
function} with regard to each model parameter \(\theta_{j}\). Therefore,
we need to calculate the \emph{partial derivative} with respect to
\(\theta_{j}\). The equation looks something like this :

\[\frac{\partial}{\partial \theta_{j}}\mathbf{MSE}\Big( \theta \Big) = \frac{2}{m} \sum_{i=1}^{m} \Bigg( \theta^{\intercal}\mathbf{x}^{(i)} - y^{(i)} \Bigg)x_{j}^{(i)}\]

Gradient vector of the cost function :

\[\nabla_{\theta} \mathbf{MSE}\Big( \theta \Big) = \begin{pmatrix} \frac{\partial}{\partial \theta_{0}}\mathbf{MSE}\Big( \theta \Big)\\ \frac{\partial}{\partial \theta_{1}}\mathbf{MSE}\Big( \theta \Big)\\ \vdots \\ \frac{\partial}{\partial \theta_{n}}\mathbf{MSE}\Big( \theta \Big) \end{pmatrix} = \frac{2}{m} \mathbf{X}^{\intercal}\Big( \mathbf{X}\theta - y \Big)\]

The only problem(or benefit, as you see it) of this algorithm is that it
involves calculations over the full training set \(\mathbf{X}\) at each
step of \emph{Gradient Descent}. That is why it is called the
\emph{Batch/Full Gradient Descent}. This makes this algorithm terribly
slow but kind of accurate.

\textbf{However}, Gradient Descent scales well with the number of
features. Training a \emph{Linear Regression} model when there are a lot
of features is much faster using \emph{Gradient Descent} than using the
\emph{Normal Equation} or \emph{SVD decomposition} :)

Finding the gradient, gives us the uphill direction, so quote trivially,
the \emph{Gradient Descent Step} looks like this :

\[\theta^{(next)} = \theta - \eta\nabla_{\theta} \mathbf{MSE}\Big( \theta \Big)\]

Here, \(\eta\) is the learning rate of the model.

Now let's try the same example we used with \emph{Normal Equation} :

    \begin{tcolorbox}[breakable, size=fbox, boxrule=1pt, pad at break*=1mm,colback=cellbackground, colframe=cellborder]
\prompt{In}{incolor}{7}{\boxspacing}
\begin{Verbatim}[commandchars=\\\{\}]
\PY{n}{learning\PYZus{}rate} \PY{o}{=} \PY{l+m+mf}{0.1}
\PY{n}{n\PYZus{}iterations} \PY{o}{=} \PY{l+m+mi}{1000}
\PY{n}{m} \PY{o}{=} \PY{l+m+mi}{100}

\PY{n}{theta} \PY{o}{=} \PY{n}{np}\PY{o}{.}\PY{n}{random}\PY{o}{.}\PY{n}{randn}\PY{p}{(}\PY{l+m+mi}{2}\PY{p}{,}\PY{l+m+mi}{1}\PY{p}{)} \PY{c+c1}{\PYZsh{}Random initialization to start with}

\PY{k}{for} \PY{n}{i} \PY{o+ow}{in} \PY{n+nb}{range}\PY{p}{(}\PY{n}{n\PYZus{}iterations}\PY{p}{)}\PY{p}{:}
    \PY{n}{gradients} \PY{o}{=} \PY{l+m+mi}{2}\PY{o}{/}\PY{n}{m} \PY{o}{*} \PY{n}{X\PYZus{}b}\PY{o}{.}\PY{n}{T}\PY{o}{.}\PY{n}{dot}\PY{p}{(}\PY{n}{X\PYZus{}b}\PY{o}{.}\PY{n}{dot}\PY{p}{(}\PY{n}{theta}\PY{p}{)} \PY{o}{\PYZhy{}} \PY{n}{y}\PY{p}{)}
    \PY{n}{theta} \PY{o}{=} \PY{n}{theta} \PY{o}{\PYZhy{}} \PY{n}{learning\PYZus{}rate}\PY{o}{*}\PY{n}{gradients}
\end{Verbatim}
\end{tcolorbox}

    The final \(\theta\) we got:

    \begin{tcolorbox}[breakable, size=fbox, boxrule=1pt, pad at break*=1mm,colback=cellbackground, colframe=cellborder]
\prompt{In}{incolor}{8}{\boxspacing}
\begin{Verbatim}[commandchars=\\\{\}]
\PY{n}{theta}
\end{Verbatim}
\end{tcolorbox}

            \begin{tcolorbox}[breakable, size=fbox, boxrule=.5pt, pad at break*=1mm, opacityfill=0]
\prompt{Out}{outcolor}{8}{\boxspacing}
\begin{Verbatim}[commandchars=\\\{\}]
array([[4.80833324],
       [7.12658528]])
\end{Verbatim}
\end{tcolorbox}
        
    That's the same as what the Normal Equation found ! But the case would
be different if we had used a different \emph{learning rate}.

    \begin{center}
    \adjustimage{max size={0.9\linewidth}{0.9\paperheight}}{download2.png}
    \end{center}
    { \hspace*{\fill} \\}

If its too low, then it'll take a lot of time to reach the optimal
solution, whereas if its too high, the algorithm diverges and we won't
even reach the solution. We can use \texttt{Scikit\ GridSearchCV} for
finding the optimal hyperparameter.

For the number of iterations : Best practice is to set a very large
number of iterations, and to interrupt the algorithm, when the
\textbf{gradient vector} becomes smaller than a certain
\emph{tolerance}(\(\epsilon\)).

    \hypertarget{stochastic-gradient-descent}{%
\subsection{Stochastic Gradient
Descent}\label{stochastic-gradient-descent}}

``Stochastic'' means ``random''. \emph{Stochastic Gradient Descent}
picks a random instance in the training set at every step and computes
the gradients based only on that single instance. Obviously, working on
a single instance at a time makes the algorithm much faster because it
has very little data to manipulate at every iteration.

    \begin{center}
    \adjustimage{max size={0.6\linewidth}{0.9\paperheight}}{download3.png}
    \end{center}
    { \hspace*{\fill} \\}

Due to its randomness, \textbf{SGD} has a better chance of finding the
global minima, than \emph{Batch Gradient Descent}. So, the randomness is
good for escaping the local optima, but bad because it can never settle
at the minima. One solution would be to gradually decrease the learning
rate. The steps start out large, then it dies down and settles at the
global minima. This process is also known as \textbf{\emph{simulated
annealing}}. the function which determines the learning rate at each
iteration is called the \textbf{\emph{learning schedule}}.

Code implementation of \emph{SGD} :

    \begin{tcolorbox}[breakable, size=fbox, boxrule=1pt, pad at break*=1mm,colback=cellbackground, colframe=cellborder]
\prompt{In}{incolor}{9}{\boxspacing}
\begin{Verbatim}[commandchars=\\\{\}]
\PY{n}{n\PYZus{}epochs} \PY{o}{=} \PY{l+m+mi}{50}
\PY{n}{t0}\PY{p}{,} \PY{n}{t1} \PY{o}{=} \PY{l+m+mi}{5}\PY{p}{,} \PY{l+m+mi}{50} \PY{c+c1}{\PYZsh{} Learning Schedule Hyperparameters}

\PY{k}{def} \PY{n+nf}{learning\PYZus{}schedule}\PY{p}{(}\PY{n}{t}\PY{p}{)}\PY{p}{:}
    \PY{k}{return} \PY{n}{t0} \PY{o}{/} \PY{p}{(}\PY{n}{t} \PY{o}{+} \PY{n}{t1}\PY{p}{)}

\PY{n}{theta} \PY{o}{=} \PY{n}{np}\PY{o}{.}\PY{n}{random}\PY{o}{.}\PY{n}{randn}\PY{p}{(}\PY{l+m+mi}{2}\PY{p}{,}\PY{l+m+mi}{1}\PY{p}{)} \PY{c+c1}{\PYZsh{} random initialization}

\PY{n}{fig} \PY{o}{=} \PY{n}{plt}\PY{o}{.}\PY{n}{figure}\PY{p}{(}\PY{p}{)}
\PY{n}{ax} \PY{o}{=} \PY{n}{fig}\PY{o}{.}\PY{n}{add\PYZus{}subplot}\PY{p}{(}\PY{l+m+mi}{111}\PY{p}{)}

\PY{k}{for} \PY{n}{epoch} \PY{o+ow}{in} \PY{n+nb}{range}\PY{p}{(}\PY{n}{n\PYZus{}epochs}\PY{p}{)}\PY{p}{:}
    \PY{k}{for} \PY{n}{i} \PY{o+ow}{in} \PY{n+nb}{range}\PY{p}{(}\PY{n}{m}\PY{p}{)}\PY{p}{:}
        \PY{n}{random\PYZus{}index} \PY{o}{=} \PY{n}{np}\PY{o}{.}\PY{n}{random}\PY{o}{.}\PY{n}{randint}\PY{p}{(}\PY{n}{m}\PY{p}{)}
        \PY{n}{xi} \PY{o}{=} \PY{n}{X\PYZus{}b}\PY{p}{[}\PY{n}{random\PYZus{}index}\PY{p}{:}\PY{n}{random\PYZus{}index}\PY{o}{+}\PY{l+m+mi}{1}\PY{p}{]}
        \PY{n}{yi} \PY{o}{=} \PY{n}{y}\PY{p}{[}\PY{n}{random\PYZus{}index}\PY{p}{:}\PY{n}{random\PYZus{}index}\PY{o}{+}\PY{l+m+mi}{1}\PY{p}{]}
        \PY{n}{gradients} \PY{o}{=} \PY{l+m+mi}{2} \PY{o}{*} \PY{n}{xi}\PY{o}{.}\PY{n}{T}\PY{o}{.}\PY{n}{dot}\PY{p}{(}\PY{n}{xi}\PY{o}{.}\PY{n}{dot}\PY{p}{(}\PY{n}{theta}\PY{p}{)} \PY{o}{\PYZhy{}} \PY{n}{yi}\PY{p}{)}
        \PY{n}{eta} \PY{o}{=} \PY{n}{learning\PYZus{}schedule}\PY{p}{(}\PY{n}{epoch} \PY{o}{*} \PY{n}{m} \PY{o}{+} \PY{n}{i}\PY{p}{)}
        \PY{n}{theta} \PY{o}{=} \PY{n}{theta} \PY{o}{\PYZhy{}} \PY{n}{eta} \PY{o}{*} \PY{n}{gradients}
        
    \PY{k}{if} \PY{n}{epoch} \PY{o}{\PYZlt{}} \PY{l+m+mi}{20} \PY{p}{:}
        \PY{n}{X\PYZus{}new} \PY{o}{=} \PY{n}{np}\PY{o}{.}\PY{n}{array}\PY{p}{(}\PY{p}{[}\PY{p}{[}\PY{l+m+mi}{0}\PY{p}{]}\PY{p}{,} \PY{p}{[}\PY{l+m+mi}{2}\PY{p}{]}\PY{p}{]}\PY{p}{)}
        \PY{n}{X\PYZus{}new\PYZus{}b} \PY{o}{=} \PY{n}{np}\PY{o}{.}\PY{n}{c\PYZus{}}\PY{p}{[}\PY{n}{np}\PY{o}{.}\PY{n}{ones}\PY{p}{(}\PY{p}{(}\PY{l+m+mi}{2}\PY{p}{,} \PY{l+m+mi}{1}\PY{p}{)}\PY{p}{)}\PY{p}{,} \PY{n}{X\PYZus{}new}\PY{p}{]}
        \PY{n}{y\PYZus{}predict} \PY{o}{=} \PY{n}{X\PYZus{}new\PYZus{}b}\PY{o}{.}\PY{n}{dot}\PY{p}{(}\PY{n}{theta}\PY{p}{)}
        \PY{k}{if} \PY{n}{epoch} \PY{o}{==} \PY{l+m+mi}{1}\PY{p}{:}
            \PY{n}{ax}\PY{o}{.}\PY{n}{plot}\PY{p}{(}\PY{n}{X\PYZus{}new}\PY{p}{,} \PY{n}{y\PYZus{}predict}\PY{p}{,} \PY{l+s+s2}{\PYZdq{}}\PY{l+s+s2}{b\PYZhy{}}\PY{l+s+s2}{\PYZdq{}}\PY{p}{,} \PY{n}{linewidth} \PY{o}{=} \PY{l+m+mi}{1}\PY{p}{,} \PY{n}{label} \PY{o}{=} \PY{l+s+s2}{\PYZdq{}}\PY{l+s+s2}{Learning Curve}\PY{l+s+s2}{\PYZdq{}}\PY{p}{)}
        \PY{k}{else} \PY{p}{:}
            \PY{n}{ax}\PY{o}{.}\PY{n}{plot}\PY{p}{(}\PY{n}{X\PYZus{}new}\PY{p}{,} \PY{n}{y\PYZus{}predict}\PY{p}{,} \PY{l+s+s2}{\PYZdq{}}\PY{l+s+s2}{b\PYZhy{}}\PY{l+s+s2}{\PYZdq{}}\PY{p}{,} \PY{n}{linewidth} \PY{o}{=} \PY{l+m+mi}{1}\PY{p}{)}

\PY{n}{ratio} \PY{o}{=} \PY{l+m+mf}{0.6}
\PY{n}{xleft}\PY{p}{,} \PY{n}{xright} \PY{o}{=} \PY{n}{ax}\PY{o}{.}\PY{n}{get\PYZus{}xlim}\PY{p}{(}\PY{p}{)}
\PY{n}{ybottom}\PY{p}{,} \PY{n}{ytop} \PY{o}{=} \PY{n}{ax}\PY{o}{.}\PY{n}{get\PYZus{}ylim}\PY{p}{(}\PY{p}{)}
\PY{n}{ax}\PY{o}{.}\PY{n}{set\PYZus{}aspect}\PY{p}{(}\PY{n+nb}{abs}\PY{p}{(}\PY{p}{(}\PY{n}{xright}\PY{o}{\PYZhy{}}\PY{n}{xleft}\PY{p}{)}\PY{o}{/}\PY{p}{(}\PY{n}{ybottom}\PY{o}{\PYZhy{}}\PY{n}{ytop}\PY{p}{)}\PY{p}{)}\PY{o}{*}\PY{n}{ratio}\PY{p}{)}

\PY{n}{plt}\PY{o}{.}\PY{n}{xlabel}\PY{p}{(}\PY{l+s+s2}{\PYZdq{}}\PY{l+s+s2}{X}\PY{l+s+s2}{\PYZdq{}}\PY{p}{)}
\PY{n}{plt}\PY{o}{.}\PY{n}{ylabel}\PY{p}{(}\PY{l+s+s2}{\PYZdq{}}\PY{l+s+s2}{y}\PY{l+s+s2}{\PYZdq{}}\PY{p}{)}
\PY{n}{plt}\PY{o}{.}\PY{n}{xlim}\PY{p}{(}\PY{l+m+mf}{0.00}\PY{p}{,}\PY{l+m+mf}{2.00}\PY{p}{)}
\PY{n}{plt}\PY{o}{.}\PY{n}{plot}\PY{p}{(}\PY{n}{X}\PY{p}{,} \PY{n}{y}\PY{p}{,} \PY{l+s+s2}{\PYZdq{}}\PY{l+s+s2}{ro}\PY{l+s+s2}{\PYZdq{}}\PY{p}{,} \PY{n}{markersize} \PY{o}{=} \PY{l+m+mi}{3}\PY{p}{,} \PY{n}{label} \PY{o}{=} \PY{l+s+s2}{\PYZdq{}}\PY{l+s+s2}{Scattered Dataset}\PY{l+s+s2}{\PYZdq{}}\PY{p}{)}
\PY{n}{plt}\PY{o}{.}\PY{n}{grid}\PY{p}{(}\PY{k+kc}{True}\PY{p}{)}
\PY{n}{plt}\PY{o}{.}\PY{n}{legend}\PY{p}{(}\PY{n}{loc}\PY{o}{=}\PY{l+s+s2}{\PYZdq{}}\PY{l+s+s2}{lower right}\PY{l+s+s2}{\PYZdq{}}\PY{p}{)}
\PY{n}{plt}\PY{o}{.}\PY{n}{title}\PY{p}{(}\PY{l+s+s2}{\PYZdq{}}\PY{l+s+s2}{First 20 steps of SGD}\PY{l+s+s2}{\PYZdq{}}\PY{p}{)}
\PY{n}{plt}\PY{o}{.}\PY{n}{show}\PY{p}{(}\PY{p}{)}
\end{Verbatim}
\end{tcolorbox}

    \begin{center}
    \adjustimage{max size={0.7\linewidth}{0.9\paperheight}}{output_22_0.png}
    \end{center}
    { \hspace*{\fill} \\}
    
    By convention we iterate by rounds of \(m\) iterations; each round is
called an \emph{epoch}. While the \emph{Batch Gradient Descent code}
iterated \textbf{1000 times} through the whole training set, this code
goes through the training set only \textbf{50 times} and reaches an
acceptable solution :

    \begin{tcolorbox}[breakable, size=fbox, boxrule=1pt, pad at break*=1mm,colback=cellbackground, colframe=cellborder]
\prompt{In}{incolor}{10}{\boxspacing}
\begin{Verbatim}[commandchars=\\\{\}]
\PY{n}{theta}
\end{Verbatim}
\end{tcolorbox}

            \begin{tcolorbox}[breakable, size=fbox, boxrule=.5pt, pad at break*=1mm, opacityfill=0]
\prompt{Out}{outcolor}{10}{\boxspacing}
\begin{Verbatim}[commandchars=\\\{\}]
array([[4.81314772],
       [7.16962854]])
\end{Verbatim}
\end{tcolorbox}
        
    \hypertarget{mini-batch-gradient-descent}{%
\subsection{Mini-batch Gradient
Descent}\label{mini-batch-gradient-descent}}

In this algorithm, at each step, instead of computing the gradients
based on the full training set (as in \emph{Batch GD}) or based on just
one instance (as in \emph{Stochastic GD}), \emph{Mini-batch GD} computes
the gradients on small random sets of instances called
\textbf{mini-batches}.

The algorithm's progress in parameter space is less erratic than with
\emph{Stochastic GD}, especially with fairly large mini-batches but both
\emph{Stochastic GD and Mini-batch GD} will reach the minimum if you use
a good learning schedule.

    \hypertarget{polynomial-regression}{%
\section{Polynomial Regression}\label{polynomial-regression}}

We can actually use a linear model to fit non-linear data. A simple way
to do this is to add powers of each feature as new features, then train
a linear model on this extended set of features. This technique is
called \textbf{\emph{Polynomial Regression}}.

Let's generate some noise in some non-linear quadratic data :

    \begin{tcolorbox}[breakable, size=fbox, boxrule=1pt, pad at break*=1mm,colback=cellbackground, colframe=cellborder]
\prompt{In}{incolor}{11}{\boxspacing}
\begin{Verbatim}[commandchars=\\\{\}]
\PY{n}{m} \PY{o}{=} \PY{l+m+mi}{100}
\PY{n}{X} \PY{o}{=} \PY{l+m+mi}{6} \PY{o}{*} \PY{n}{np}\PY{o}{.}\PY{n}{random}\PY{o}{.}\PY{n}{rand}\PY{p}{(}\PY{n}{m}\PY{p}{,} \PY{l+m+mi}{1}\PY{p}{)} \PY{o}{\PYZhy{}} \PY{l+m+mi}{3}
\PY{n}{y} \PY{o}{=} \PY{l+m+mf}{0.5} \PY{o}{*} \PY{n}{X}\PY{o}{*}\PY{o}{*}\PY{l+m+mi}{2} \PY{o}{+} \PY{n}{X} \PY{o}{+} \PY{l+m+mi}{2} \PY{o}{+} \PY{n}{np}\PY{o}{.}\PY{n}{random}\PY{o}{.}\PY{n}{randn}\PY{p}{(}\PY{n}{m}\PY{p}{,} \PY{l+m+mi}{1}\PY{p}{)}
\end{Verbatim}
\end{tcolorbox}

    Now, we'll use \texttt{Scikit-Learn\ PolynomialFeatures} class to
transform our training data :

    \begin{tcolorbox}[breakable, size=fbox, boxrule=1pt, pad at break*=1mm,colback=cellbackground, colframe=cellborder]
\prompt{In}{incolor}{12}{\boxspacing}
\begin{Verbatim}[commandchars=\\\{\}]
\PY{k+kn}{from} \PY{n+nn}{sklearn}\PY{n+nn}{.}\PY{n+nn}{preprocessing} \PY{k+kn}{import} \PY{n}{PolynomialFeatures}
\PY{n}{poly\PYZus{}features} \PY{o}{=} \PY{n}{PolynomialFeatures}\PY{p}{(}\PY{n}{degree}\PY{o}{=}\PY{l+m+mi}{2}\PY{p}{,} \PY{n}{include\PYZus{}bias}\PY{o}{=}\PY{k+kc}{False}\PY{p}{)}
\PY{n}{X\PYZus{}poly} \PY{o}{=} \PY{n}{poly\PYZus{}features}\PY{o}{.}\PY{n}{fit\PYZus{}transform}\PY{p}{(}\PY{n}{X}\PY{p}{)}
\PY{n}{X\PYZus{}poly}\PY{p}{[}\PY{l+m+mi}{0}\PY{p}{]}
\end{Verbatim}
\end{tcolorbox}

            \begin{tcolorbox}[breakable, size=fbox, boxrule=.5pt, pad at break*=1mm, opacityfill=0]
\prompt{Out}{outcolor}{12}{\boxspacing}
\begin{Verbatim}[commandchars=\\\{\}]
array([2.86334939, 8.19876974])
\end{Verbatim}
\end{tcolorbox}
        
    Now let's fit a \textbf{\emph{Linear Regression}} model to this extended
training data( we are using \texttt{Scikit-Learn\ LinearRegression} here
:

    \begin{tcolorbox}[breakable, size=fbox, boxrule=1pt, pad at break*=1mm,colback=cellbackground, colframe=cellborder]
\prompt{In}{incolor}{13}{\boxspacing}
\begin{Verbatim}[commandchars=\\\{\}]
\PY{k+kn}{from} \PY{n+nn}{sklearn}\PY{n+nn}{.}\PY{n+nn}{linear\PYZus{}model} \PY{k+kn}{import} \PY{n}{LinearRegression}
\PY{n}{lin\PYZus{}reg} \PY{o}{=} \PY{n}{LinearRegression}\PY{p}{(}\PY{p}{)}
\PY{n}{lin\PYZus{}reg}\PY{o}{.}\PY{n}{fit}\PY{p}{(}\PY{n}{X\PYZus{}poly}\PY{p}{,} \PY{n}{y}\PY{p}{)}
\PY{n}{lin\PYZus{}reg}\PY{o}{.}\PY{n}{intercept\PYZus{}}\PY{p}{,} \PY{n}{lin\PYZus{}reg}\PY{o}{.}\PY{n}{coef\PYZus{}}
\end{Verbatim}
\end{tcolorbox}

            \begin{tcolorbox}[breakable, size=fbox, boxrule=.5pt, pad at break*=1mm, opacityfill=0]
\prompt{Out}{outcolor}{13}{\boxspacing}
\begin{Verbatim}[commandchars=\\\{\}]
(array([1.98031152]), array([[1.06214383, 0.53454417]]))
\end{Verbatim}
\end{tcolorbox}
        
    Let's plot and see what we get !

    \begin{tcolorbox}[breakable, size=fbox, boxrule=1pt, pad at break*=1mm,colback=cellbackground, colframe=cellborder]
\prompt{In}{incolor}{14}{\boxspacing}
\begin{Verbatim}[commandchars=\\\{\}]
\PY{n}{X\PYZus{}new} \PY{o}{=} \PY{n}{np}\PY{o}{.}\PY{n}{arange}\PY{p}{(}\PY{o}{\PYZhy{}}\PY{l+m+mi}{3}\PY{p}{,}\PY{l+m+mf}{3.2}\PY{p}{,}\PY{l+m+mf}{0.1}\PY{p}{)}
\PY{n}{X\PYZus{}new\PYZus{}b} \PY{o}{=} \PY{n}{np}\PY{o}{.}\PY{n}{c\PYZus{}}\PY{p}{[}\PY{n}{X\PYZus{}new}\PY{p}{,} \PY{n}{X\PYZus{}new}\PY{o}{*}\PY{o}{*}\PY{l+m+mi}{2}\PY{p}{]}
\PY{n}{y\PYZus{}predict} \PY{o}{=}   \PY{n}{np}\PY{o}{.}\PY{n}{c\PYZus{}}\PY{p}{[}\PY{n}{np}\PY{o}{.}\PY{n}{ones}\PY{p}{(}\PY{p}{(}\PY{l+m+mi}{62}\PY{p}{,}\PY{l+m+mi}{1}\PY{p}{)}\PY{p}{)}\PY{o}{.}\PY{n}{dot}\PY{p}{(}\PY{n}{lin\PYZus{}reg}\PY{o}{.}\PY{n}{intercept\PYZus{}}\PY{p}{)} \PY{o}{+} \PY{n}{X\PYZus{}new}\PY{o}{*}\PY{n}{lin\PYZus{}reg}\PY{o}{.}\PY{n}{coef\PYZus{}}\PY{p}{[}\PY{p}{[}\PY{l+m+mi}{0}\PY{p}{]}\PY{p}{[}\PY{l+m+mi}{0}\PY{p}{]}\PY{p}{]}\PY{p}{[}\PY{l+m+mi}{0}\PY{p}{]} \PY{o}{+} \PY{p}{(}\PY{n}{X\PYZus{}new}\PY{o}{*}\PY{o}{*}\PY{l+m+mi}{2}\PY{p}{)}\PY{o}{*}\PY{n}{lin\PYZus{}reg}\PY{o}{.}\PY{n}{coef\PYZus{}}\PY{p}{[}\PY{p}{[}\PY{l+m+mi}{0}\PY{p}{]}\PY{p}{[}\PY{l+m+mi}{0}\PY{p}{]}\PY{p}{]}\PY{p}{[}\PY{l+m+mi}{1}\PY{p}{]}\PY{p}{]}

\PY{n}{plt}\PY{o}{.}\PY{n}{plot}\PY{p}{(}\PY{n}{X}\PY{p}{,} \PY{n}{y}\PY{p}{,} \PY{l+s+s2}{\PYZdq{}}\PY{l+s+s2}{bo}\PY{l+s+s2}{\PYZdq{}}\PY{p}{,} \PY{n}{markersize} \PY{o}{=} \PY{l+m+mi}{2}\PY{p}{,} \PY{n}{label} \PY{o}{=} \PY{l+s+s2}{\PYZdq{}}\PY{l+s+s2}{Scattered Dataset}\PY{l+s+s2}{\PYZdq{}}\PY{p}{)}
\PY{n}{plt}\PY{o}{.}\PY{n}{plot}\PY{p}{(}\PY{n}{X\PYZus{}new}\PY{p}{,} \PY{n}{y\PYZus{}predict}\PY{p}{,} \PY{l+s+s2}{\PYZdq{}}\PY{l+s+s2}{r\PYZhy{}}\PY{l+s+s2}{\PYZdq{}}\PY{p}{,} \PY{n}{label} \PY{o}{=} \PY{l+s+s2}{\PYZdq{}}\PY{l+s+s2}{Predicted Curve}\PY{l+s+s2}{\PYZdq{}}\PY{p}{)}
\PY{n}{plt}\PY{o}{.}\PY{n}{title}\PY{p}{(}\PY{l+s+s2}{\PYZdq{}}\PY{l+s+s2}{Polynomial Regression}\PY{l+s+s2}{\PYZdq{}}\PY{p}{)}
\PY{n}{plt}\PY{o}{.}\PY{n}{grid}\PY{p}{(}\PY{l+s+s2}{\PYZdq{}}\PY{l+s+s2}{True}\PY{l+s+s2}{\PYZdq{}}\PY{p}{)}
\PY{n}{plt}\PY{o}{.}\PY{n}{xlim}\PY{p}{(}\PY{o}{\PYZhy{}}\PY{l+m+mi}{3}\PY{p}{,}\PY{l+m+mi}{3}\PY{p}{)}
\PY{n}{plt}\PY{o}{.}\PY{n}{xlabel}\PY{p}{(}\PY{l+s+s2}{\PYZdq{}}\PY{l+s+s2}{X}\PY{l+s+s2}{\PYZdq{}}\PY{p}{)}
\PY{n}{plt}\PY{o}{.}\PY{n}{ylabel}\PY{p}{(}\PY{l+s+s2}{\PYZdq{}}\PY{l+s+s2}{y}\PY{l+s+s2}{\PYZdq{}}\PY{p}{)}
\PY{n}{plt}\PY{o}{.}\PY{n}{legend}\PY{p}{(}\PY{n}{loc}\PY{o}{=}\PY{l+s+s2}{\PYZdq{}}\PY{l+s+s2}{lower right}\PY{l+s+s2}{\PYZdq{}}\PY{p}{)}
\PY{n}{plt}\PY{o}{.}\PY{n}{show}\PY{p}{(}\PY{p}{)}
\end{Verbatim}
\end{tcolorbox}

    \begin{center}
    \adjustimage{max size={0.8\linewidth}{0.9\paperheight}}{output_33_0.png}
    \end{center}
    { \hspace*{\fill} \\}
    
    \hypertarget{learning-curves}{%
\section{Learning Curves}\label{learning-curves}}

If you perform high-degree Polynomial Regression, you will likely fit
the training data much better than with plain Linear Regression.

But a very-high degree polynomial has chances of severely
\emph{overfitting} the training data, while a linear model might have
chances of \emph{underfitting} the same data. So we can use
\textbf{cross-validation} to get an estimate of the model's performance.
If a model performs well on the training data but generalizes poorly
according to the cross-validation metrics, then the model is
overfitting. If it performs poorly on both, then it is underfitting.

Another way to tell it, is to look at the \emph{learning curves} : These
are plots of the model's performance on the training set and the
validation set as a function of the training set size. Let's generate
the \emph{Learning Curves} here :

    \begin{tcolorbox}[breakable, size=fbox, boxrule=1pt, pad at break*=1mm,colback=cellbackground, colframe=cellborder]
\prompt{In}{incolor}{15}{\boxspacing}
\begin{Verbatim}[commandchars=\\\{\}]
\PY{k+kn}{from} \PY{n+nn}{sklearn}\PY{n+nn}{.}\PY{n+nn}{metrics} \PY{k+kn}{import} \PY{n}{mean\PYZus{}squared\PYZus{}error}
\PY{k+kn}{from} \PY{n+nn}{sklearn}\PY{n+nn}{.}\PY{n+nn}{model\PYZus{}selection} \PY{k+kn}{import} \PY{n}{train\PYZus{}test\PYZus{}split}

\PY{c+c1}{\PYZsh{}\PYZsh{}A function that, given some training data, plots the learning curves of a model:}
\PY{k}{def} \PY{n+nf}{plot\PYZus{}learning\PYZus{}curves}\PY{p}{(}\PY{n}{model}\PY{p}{,} \PY{n}{X}\PY{p}{,} \PY{n}{y}\PY{p}{)}\PY{p}{:}
    \PY{n}{X\PYZus{}train}\PY{p}{,} \PY{n}{X\PYZus{}val}\PY{p}{,} \PY{n}{y\PYZus{}train}\PY{p}{,} \PY{n}{y\PYZus{}val} \PY{o}{=} \PY{n}{train\PYZus{}test\PYZus{}split}\PY{p}{(}\PY{n}{X}\PY{p}{,} \PY{n}{y}\PY{p}{,} \PY{n}{test\PYZus{}size}\PY{o}{=}\PY{l+m+mf}{0.2}\PY{p}{)}
    \PY{n}{train\PYZus{}errors}\PY{p}{,} \PY{n}{val\PYZus{}errors} \PY{o}{=} \PY{p}{[}\PY{p}{]}\PY{p}{,} \PY{p}{[}\PY{p}{]}
    \PY{k}{for} \PY{n}{m} \PY{o+ow}{in} \PY{n+nb}{range}\PY{p}{(}\PY{l+m+mi}{1}\PY{p}{,} \PY{n+nb}{len}\PY{p}{(}\PY{n}{X\PYZus{}train}\PY{p}{)}\PY{p}{)}\PY{p}{:}
        \PY{n}{model}\PY{o}{.}\PY{n}{fit}\PY{p}{(}\PY{n}{X\PYZus{}train}\PY{p}{[}\PY{p}{:}\PY{n}{m}\PY{p}{]}\PY{p}{,} \PY{n}{y\PYZus{}train}\PY{p}{[}\PY{p}{:}\PY{n}{m}\PY{p}{]}\PY{p}{)}
        \PY{n}{y\PYZus{}train\PYZus{}predict} \PY{o}{=} \PY{n}{model}\PY{o}{.}\PY{n}{predict}\PY{p}{(}\PY{n}{X\PYZus{}train}\PY{p}{[}\PY{p}{:}\PY{n}{m}\PY{p}{]}\PY{p}{)}
        \PY{n}{y\PYZus{}val\PYZus{}predict} \PY{o}{=} \PY{n}{model}\PY{o}{.}\PY{n}{predict}\PY{p}{(}\PY{n}{X\PYZus{}val}\PY{p}{)}
        \PY{n}{train\PYZus{}errors}\PY{o}{.}\PY{n}{append}\PY{p}{(}\PY{n}{mean\PYZus{}squared\PYZus{}error}\PY{p}{(}\PY{n}{y\PYZus{}train}\PY{p}{[}\PY{p}{:}\PY{n}{m}\PY{p}{]}\PY{p}{,}
        \PY{n}{y\PYZus{}train\PYZus{}predict}\PY{p}{)}\PY{p}{)}
        \PY{n}{val\PYZus{}errors}\PY{o}{.}\PY{n}{append}\PY{p}{(}\PY{n}{mean\PYZus{}squared\PYZus{}error}\PY{p}{(}\PY{n}{y\PYZus{}val}\PY{p}{,} \PY{n}{y\PYZus{}val\PYZus{}predict}\PY{p}{)}\PY{p}{)}
    \PY{n}{plt}\PY{o}{.}\PY{n}{plot}\PY{p}{(}\PY{n}{np}\PY{o}{.}\PY{n}{sqrt}\PY{p}{(}\PY{n}{train\PYZus{}errors}\PY{p}{)}\PY{p}{,} \PY{l+s+s2}{\PYZdq{}}\PY{l+s+s2}{g\PYZhy{}+}\PY{l+s+s2}{\PYZdq{}}\PY{p}{,} \PY{n}{linewidth}\PY{o}{=}\PY{l+m+mi}{2}\PY{p}{,} \PY{n}{label}\PY{o}{=}\PY{l+s+s2}{\PYZdq{}}\PY{l+s+s2}{train}\PY{l+s+s2}{\PYZdq{}}\PY{p}{)}
    \PY{n}{plt}\PY{o}{.}\PY{n}{plot}\PY{p}{(}\PY{n}{np}\PY{o}{.}\PY{n}{sqrt}\PY{p}{(}\PY{n}{val\PYZus{}errors}\PY{p}{)}\PY{p}{,} \PY{l+s+s2}{\PYZdq{}}\PY{l+s+s2}{r\PYZhy{}}\PY{l+s+s2}{\PYZdq{}}\PY{p}{,} \PY{n}{linewidth}\PY{o}{=}\PY{l+m+mi}{3}\PY{p}{,} \PY{n}{label}\PY{o}{=}\PY{l+s+s2}{\PYZdq{}}\PY{l+s+s2}{val}\PY{l+s+s2}{\PYZdq{}}\PY{p}{)}
    \PY{n}{plt}\PY{o}{.}\PY{n}{grid}\PY{p}{(}\PY{k+kc}{True}\PY{p}{)}
    \PY{n}{plt}\PY{o}{.}\PY{n}{legend}\PY{p}{(}\PY{n}{loc} \PY{o}{=} \PY{l+s+s2}{\PYZdq{}}\PY{l+s+s2}{upper right}\PY{l+s+s2}{\PYZdq{}}\PY{p}{)}
    \PY{n}{plt}\PY{o}{.}\PY{n}{xlabel}\PY{p}{(}\PY{l+s+s2}{\PYZdq{}}\PY{l+s+s2}{Training Set Size}\PY{l+s+s2}{\PYZdq{}}\PY{p}{)}
    \PY{n}{plt}\PY{o}{.}\PY{n}{ylabel}\PY{p}{(}\PY{l+s+s2}{\PYZdq{}}\PY{l+s+s2}{RMSE}\PY{l+s+s2}{\PYZdq{}}\PY{p}{)}
    \PY{n}{plt}\PY{o}{.}\PY{n}{title}\PY{p}{(}\PY{l+s+s2}{\PYZdq{}}\PY{l+s+s2}{Learning Curves}\PY{l+s+s2}{\PYZdq{}}\PY{p}{)}
    \PY{n}{plt}\PY{o}{.}\PY{n}{ylim}\PY{p}{(}\PY{l+m+mf}{0.00}\PY{p}{,}\PY{l+m+mf}{3.00}\PY{p}{)}
    \PY{n}{plt}\PY{o}{.}\PY{n}{show}\PY{p}{(}\PY{p}{)}
\end{Verbatim}
\end{tcolorbox}

    For a plain \emph{Linear Regression} model :

    \begin{tcolorbox}[breakable, size=fbox, boxrule=1pt, pad at break*=1mm,colback=cellbackground, colframe=cellborder]
\prompt{In}{incolor}{16}{\boxspacing}
\begin{Verbatim}[commandchars=\\\{\}]
\PY{n}{lin\PYZus{}reg} \PY{o}{=} \PY{n}{LinearRegression}\PY{p}{(}\PY{p}{)}
\PY{n}{plot\PYZus{}learning\PYZus{}curves}\PY{p}{(}\PY{n}{lin\PYZus{}reg}\PY{p}{,} \PY{n}{X}\PY{p}{,} \PY{n}{y}\PY{p}{)}
\end{Verbatim}
\end{tcolorbox}

    \begin{center}
    \adjustimage{max size={0.7\linewidth}{0.9\paperheight}}{output_37_0.png}
    \end{center}
    { \hspace*{\fill} \\}
    
    This is a typical example of underfitting. Both training and validation
curves have reached a plateau, they are close and high. Let's plot the
\emph{learning curve} for a 10th-degree polynomial :

    \begin{tcolorbox}[breakable, size=fbox, boxrule=1pt, pad at break*=1mm,colback=cellbackground, colframe=cellborder]
\prompt{In}{incolor}{17}{\boxspacing}
\begin{Verbatim}[commandchars=\\\{\}]
\PY{c+c1}{\PYZsh{}\PYZsh{}Create a pipeline for Scikit first :)}

\PY{k+kn}{from} \PY{n+nn}{sklearn}\PY{n+nn}{.}\PY{n+nn}{pipeline} \PY{k+kn}{import} \PY{n}{Pipeline}
\PY{n}{polynomial\PYZus{}regression} \PY{o}{=} \PY{n}{Pipeline}\PY{p}{(}\PY{p}{[}
                        \PY{p}{(}\PY{l+s+s2}{\PYZdq{}}\PY{l+s+s2}{poly\PYZus{}features}\PY{l+s+s2}{\PYZdq{}}\PY{p}{,} \PY{n}{PolynomialFeatures}\PY{p}{(}\PY{n}{degree}\PY{o}{=}\PY{l+m+mi}{10}\PY{p}{,}
                        \PY{n}{include\PYZus{}bias}\PY{o}{=}\PY{k+kc}{False}\PY{p}{)}\PY{p}{)}\PY{p}{,}
                        \PY{p}{(}\PY{l+s+s2}{\PYZdq{}}\PY{l+s+s2}{lin\PYZus{}reg}\PY{l+s+s2}{\PYZdq{}}\PY{p}{,} \PY{n}{LinearRegression}\PY{p}{(}\PY{p}{)}\PY{p}{)}\PY{p}{,}
                        \PY{p}{]}\PY{p}{)}

\PY{n}{plot\PYZus{}learning\PYZus{}curves}\PY{p}{(}\PY{n}{polynomial\PYZus{}regression}\PY{p}{,} \PY{n}{X}\PY{p}{,} \PY{n}{y}\PY{p}{)}
\end{Verbatim}
\end{tcolorbox}

    \begin{center}
    \adjustimage{max size={0.7\linewidth}{0.9\paperheight}}{output_39_0.png}
    \end{center}
    { \hspace*{\fill} \\}
    
    Differences :

\begin{itemize}
\item
  The error on the training data is much lower than with the Linear
  Regression model.
\item
  There's a gap between the curves, which indicates that this model is
  the heavily overfitting :(
\end{itemize}

\textbf{\emph{Note}} : Its good to notice that, as the train set size
increases the model performs much better, just like any other
overfitting scenario.

    \hypertarget{regularized-linear-models}{%
\section{Regularized Linear Models}\label{regularized-linear-models}}

The fewer degrees of freedom a model has, the harder it will be for it
to overfit the data. A simple way to regularize a polynomial model is to
reduce the number of polynomial degrees.

For a linear model, \emph{regularization} is typically achieved by
constraining the weights of the model. We will now look at \textbf{Ridge
Regression, Lasso Regression, and Elastic Net}, which implement three
different ways to constrain the weights.

    \hypertarget{ridge-regression}{%
\subsection{Ridge Regression}\label{ridge-regression}}

Ridge Regression cost function :

\[J \big( \theta \big) = \mathbf{MSE}\big( \theta \big) + \alpha\frac{1}{2}\sum_{i=1}^{n} \theta_{i}^{2}\]

Here :

\begin{itemize}
\item
  \(\alpha\frac{1}{2}\sum_{i=1}^{n} \theta_{i}^{2}\) is the
  regularization term. This forces the learning algorithm to not only
  fit the data but also keep the model weights as small as possible. We
  apply the regularization term only during the model training. Once the
  model is trained, we use the unregularized performance.
\item
  The hyperparameter \(\alpha\) controls how much we want to regularize the
  model. If \(\alpha = 0\), then \emph{Ridge Regression} is just
  \emph{Linear Regression}. If \(\alpha\) is very large, then all
  weights end up very close to zero and the result is a flat line going
  through the mean.
\end{itemize}

Ridge Regression closed-form solution :
\[\hat\theta = \big( \mathbf{X^{\intercal} X} + \alpha\mathbf{A} \big)^{-1}\mathbf{X^{\intercal}y}\]

We can perform \emph{Ridge Regression} either by computing a closed-form
equation or by performing \emph{Gradient Descent}. The pros and cons are
the same.

    \hypertarget{lasso-regression}{%
\subsection{Lasso Regression}\label{lasso-regression}}

Least Absolute Shrinkage and Selection Operator Regression (usually
simply called Lasso Regression) is another regularized version of Linear
Regression. But it uses the norm of the weight vector instead of half
the square of the norm as the regularization term.

Lasso Regression cost function :
\[J \big( \theta \big) = \mathbf{MSE}\big( \theta \big) + \alpha\sum_{i=1}^{n} |\theta_{i}|\]

An important characteristic of \emph{Lasso Regression} is that it tends
to eliminate the weights of the least important features. \emph{Lasso
Regression} automatically performs feature selection and outputs a
\emph{sparse model}.

Comparison of \textbf{Lasso and Ridge models} :

    \begin{center}
    \adjustimage{max size={0.8\linewidth}{0.9\paperheight}}{download5.png}
    \end{center}
    { \hspace*{\fill} \\}
    
    \begin{center}
    \adjustimage{max size={0.8\linewidth}{0.9\paperheight}}{download4.png}
    \end{center}
    { \hspace*{\fill} \\}

The axes represent two model parameters, and the background contours
represent different loss functions. The small white circles show the
path that Gradient Descent takes to optimize some model parameters that
were initialized.

    \hypertarget{elastic-net}{%
\subsection{Elastic Net}\label{elastic-net}}

\textbf{Elastic Net} is a middle ground between \emph{Ridge Regression}
and \emph{Lasso Regression}. The regularization term is a simple mix of
both \emph{Ridge} and \emph{Lasso's regularization terms}, and you can
control the mix ratio \emph{r}. When \emph{r = 0}, \emph{Elastic Net} is
equivalent to \emph{Ridge Regression}, and when \emph{r = 1}, it is
equivalent to \emph{Lasso Regression}.

Elastic Net cost function :

\[J \big( \theta \big) = \mathbf{MSE}\big( \theta \big) + r\alpha\sum_{i=1}^{n} |\theta_{i}| + \frac{1-r}{2}\alpha\sum_{i=1}^{n}\theta_{i}^{2}\]

    \hypertarget{early-stopping}{%
\subsection{Early Stopping}\label{early-stopping}}

A very different way to regularize iterative learning algorithms such as
Gradient Descent is to stop training as soon as the validation error
reaches a minimum. This is called \textbf{early stopping}.

As the algorithm learns, the prediction error(RMSE) on the training set
goes down, along with its prediction error on the validation set. After
a number of \emph{epochs} the \textbf{validation error stops decreasing
and starts to go back up}. This is an indication of premature
overfitting. We need to stop right here.

It functions as a simple and efficient regularization technique. We just
need to find the minima of the \emph{validation error} :)

Basic implementation of early stopping :

    \begin{tcolorbox}[breakable, size=fbox, boxrule=1pt, pad at break*=1mm,colback=cellbackground, colframe=cellborder]
\prompt{In}{incolor}{ }{\boxspacing}
\begin{Verbatim}[commandchars=\\\{\}]
\PY{k+kn}{from} \PY{n+nn}{sklearn}\PY{n+nn}{.}\PY{n+nn}{base} \PY{k+kn}{import} \PY{n}{clone}
\PY{k+kn}{from} \PY{n+nn}{sklearn}\PY{n+nn}{.}\PY{n+nn}{preprocessing} \PY{k+kn}{import} \PY{n}{StandardScaler}
\PY{c+c1}{\PYZsh{} prepare the data}
\PY{n}{poly\PYZus{}scaler} \PY{o}{=} \PY{n}{Pipeline}\PY{p}{(}\PY{p}{[}
                        \PY{p}{(}\PY{l+s+s2}{\PYZdq{}}\PY{l+s+s2}{poly\PYZus{}features}\PY{l+s+s2}{\PYZdq{}}\PY{p}{,} \PY{n}{PolynomialFeatures}\PY{p}{(}\PY{n}{degree}\PY{o}{=}\PY{l+m+mi}{90}\PY{p}{,}
                                                    \PY{n}{include\PYZus{}bias}\PY{o}{=}\PY{k+kc}{False}\PY{p}{)}\PY{p}{)}\PY{p}{,}
                        \PY{p}{(}\PY{l+s+s2}{\PYZdq{}}\PY{l+s+s2}{std\PYZus{}scaler}\PY{l+s+s2}{\PYZdq{}}\PY{p}{,} \PY{n}{StandardScaler}\PY{p}{(}\PY{p}{)}\PY{p}{)}
                        \PY{p}{]}\PY{p}{)}

\PY{n}{X\PYZus{}train\PYZus{}poly\PYZus{}scaled} \PY{o}{=} \PY{n}{poly\PYZus{}scaler}\PY{o}{.}\PY{n}{fit\PYZus{}transform}\PY{p}{(}\PY{n}{X\PYZus{}train}\PY{p}{)}
\PY{n}{X\PYZus{}val\PYZus{}poly\PYZus{}scaled} \PY{o}{=} \PY{n}{poly\PYZus{}scaler}\PY{o}{.}\PY{n}{transform}\PY{p}{(}\PY{n}{X\PYZus{}val}\PY{p}{)}

\PY{n}{sgd\PYZus{}reg} \PY{o}{=} \PY{n}{SGDRegressor}\PY{p}{(}\PY{n}{max\PYZus{}iter}\PY{o}{=}\PY{l+m+mi}{1}\PY{p}{,} \PY{n}{tol}\PY{o}{=}\PY{o}{\PYZhy{}}\PY{n}{np}\PY{o}{.}\PY{n}{infty}\PY{p}{,} \PY{n}{warm\PYZus{}start}\PY{o}{=}\PY{k+kc}{True}\PY{p}{,}
                    \PY{n}{penalty}\PY{o}{=}\PY{k+kc}{None}\PY{p}{,} \PY{n}{learning\PYZus{}rate}\PY{o}{=}\PY{l+s+s2}{\PYZdq{}}\PY{l+s+s2}{constant}\PY{l+s+s2}{\PYZdq{}}\PY{p}{,} \PY{n}{eta0}\PY{o}{=}\PY{l+m+mf}{0.0005}\PY{p}{)}

\PY{n}{minimum\PYZus{}val\PYZus{}error} \PY{o}{=} \PY{n+nb}{float}\PY{p}{(}\PY{l+s+s2}{\PYZdq{}}\PY{l+s+s2}{inf}\PY{l+s+s2}{\PYZdq{}}\PY{p}{)}
\PY{n}{best\PYZus{}epoch} \PY{o}{=} \PY{k+kc}{None}
\PY{n}{best\PYZus{}model} \PY{o}{=} \PY{k+kc}{None}

\PY{k}{for} \PY{n}{epoch} \PY{o+ow}{in} \PY{n+nb}{range}\PY{p}{(}\PY{l+m+mi}{1000}\PY{p}{)}\PY{p}{:}
    \PY{n}{sgd\PYZus{}reg}\PY{o}{.}\PY{n}{fit}\PY{p}{(}\PY{n}{X\PYZus{}train\PYZus{}poly\PYZus{}scaled}\PY{p}{,} \PY{n}{y\PYZus{}train}\PY{p}{)} \PY{c+c1}{\PYZsh{} continues where it left off}
    \PY{n}{y\PYZus{}val\PYZus{}predict} \PY{o}{=} \PY{n}{sgd\PYZus{}reg}\PY{o}{.}\PY{n}{predict}\PY{p}{(}\PY{n}{X\PYZus{}val\PYZus{}poly\PYZus{}scaled}\PY{p}{)}
    \PY{n}{val\PYZus{}error} \PY{o}{=} \PY{n}{mean\PYZus{}squared\PYZus{}error}\PY{p}{(}\PY{n}{y\PYZus{}val}\PY{p}{,} \PY{n}{y\PYZus{}val\PYZus{}predict}\PY{p}{)}
    
    \PY{k}{if} \PY{n}{val\PYZus{}error} \PY{o}{\PYZlt{}} \PY{n}{minimum\PYZus{}val\PYZus{}error}\PY{p}{:}
        \PY{n}{minimum\PYZus{}val\PYZus{}error} \PY{o}{=} \PY{n}{val\PYZus{}error}
        \PY{n}{best\PYZus{}epoch} \PY{o}{=} \PY{n}{epoch}
        \PY{n}{best\PYZus{}model} \PY{o}{=} \PY{n}{clone}\PY{p}{(}\PY{n}{sgd\PYZus{}reg}\PY{p}{)}
\end{Verbatim}
\end{tcolorbox}

    With \texttt{warm\_start=True}, when the \texttt{fit()} method is called
it continues training where it left off, instead of restarting from
scratch.

    \hypertarget{logistic-regression}{%
\section{Logistic Regression}\label{logistic-regression}}

\emph{Logistic Regression} is commonly used to estimate the probability
that an instance belongs to a particular class. Therefore, it is a
binary classifier.

Just like a Linear Regression model, a \emph{Logistic Regression} model,
computes a weighted sum of the input features and it outputs the
\emph{logistic} of this result.

Logistic Regression model \textbf{estimated probability} :

\[\hat{p} = h_{\theta}(\mathbf{x}) = \sigma \big( \mathbf{x^{\intercal}}\theta \big)\]

\(\sigma\) is the \emph{sigmoid function}. It takes the
\(\mathbf{x^{\intercal}}\theta\) and squeezes its value to a number
between 0 and 1.

Notice that \(\sigma(t) < 0.5\) when \(t < 0\), and \(\sigma(t) \geq 0.5\)
when \(t \geq 0\), so a \emph{Logistic Regression} model predicts \(1\) if
\(\mathbf{x^{\intercal}}\theta\) is positive and 0 if it is negative.

\hypertarget{training-and-cost-function}{%
\subsection{Training and Cost
Function}\label{training-and-cost-function}}

The objective of training is to set the parameter vector \(\theta\) so
that the model estimates high probabilities for positive instances (y =
1) and low probabilities for negative instances (y = 0).

Such a cost function which implements this concept for a single training
instance :

\[c(\theta) = \begin{cases} \mathbf{-log}(\hat{p}) & \text{if y = 1}\\ \mathbf{-log}(1-\hat{p}) & \text{if y = 0}\\ \end{cases}\]

The cost function over the whole training set is the average cost over
all training instances :

\[J(\theta) = -\frac{1}{m}\sum_{i=1}^{m} \Big[ y^{(i)}log(\hat{p}^{(i)}) + (1 - y^{(i)})log(1-\hat{p}^{(i)})\Big]\]

This is known as, \emph{Logistic Regression cost function(log loss)}.

Well, here we doon't have a generalised Normal Equation, but we can
easily check that the cost function(log loss) is actually convex, so we
can apply our second option, the \textbf{Gradient Descent}.

The partial derivatives of the cost function with respect to the
\(j^{th}\) model parameter is :

\[\frac{\partial}{\partial \theta_{j}}\mathbf{J(}\theta) = \frac{1}{m} \sum_{i=1}^{m}\Big(\sigma(\theta^{\intercal}\mathbf{x^{(i)}}) - y^{(i)}\Big)x_{j}^{(i)}\]

That's it, we can now apply \emph{Batch, Stochastic or Mini-Batch
Gradient Descent} to use this \emph{Logistic Regression model} :)

    \hypertarget{softmax-regression}{%
\subsection{Softmax Regression}\label{softmax-regression}}

The Logistic Regression model can be generalized to support multiple
classes directly, without having to train and combine multiple binary
classifiers. This is called \textbf{Softmax Regression, or Multinomial
Logistic Regression.}

The idea is simple: when given an instance \(x\), the Softmax Regression
model first computes a score \(s_{k}(x)\) for each class \(k\), then
estimates the probability of each class by applying the \textbf{softmax
function} (also called the \textbf{normalized exponential}) to the
scores.

Softmax score for \(k^{th}\) class :

\[ s_{k}(x) = \mathbf{x^\intercal}\theta^{(k)} \]

Note that each class has its own dedicated parameter vector
\(\theta^{(k)}\) . All these vectors are typically stored as rows in a
\textbf{parameter matrix} \(\Theta\).

We can estimate the probability that the instance belongs to class \(k\)
by running the scores through the \textbf{softmax function}.

\[\hat{p}_{k} = \sigma(\mathbf{s(x)}) = \frac{\mathbf{exp}(s_k(\mathbf{x}))}{\sum_{j=1}^{K}\mathbf{exp}(s_j(\mathbf{x}))}\]

Just like the \emph{Logistic Regression} classifier, the \textbf{Softmax
Regression} classifier predicts the class with the highest estimated
probability.

    \hypertarget{so-a-overview-of-things-covered-so-far}{%
\section{So a overview of things covered so far
:}\label{so-a-overview-of-things-covered-so-far}}

We saw the working of quite a few, Machine Learning \textbf{Regressors}
and \textbf{Classifiers}. We will now move on to more interesting stuff
\(:)\) . But before that let's summarize a bit. Firstly we started off
with \emph{Linear Regression}, and we saw that using \emph{Normal
Equation} was a pretty slow idea, so we moved towards \emph{SVD
decomposition}, and it was faster but it wasn't the thing that we
wanted ! Then we looked at a much better optimization algorithm called
the \emph{Gradient Descent}.

After we had solved the \emph{speed} issue concerning \emph{Linear
Regression} we moved on to, the predictions and fitting issues it has.
On the way of doing so we looked at an obvious choice of
\emph{Polynomial Regression}. Then we explored a new judging meter called
\emph{Learning Curves} and found that a high degree polynomial might
overfit the training data, while a linear model might heavily underfit
the same data. So an obvious choice to determine the correct polynomial
degree was to limit its degrees of freedom and we did that using a class
of models called the \emph{Regularized Linear Models}.

Then, we moved on to a commonly used binary classifier \emph{Logistic
Regression}, and its generalized version \emph{Softmax Regression}. Now
let's take a look at \emph{Decision Trees and Random Forests} and how
those compare with the logistic regression model we just studied.

    \hypertarget{decision-tree-an-overview}{%
\section{Decision Tree : An
Overview}\label{decision-tree-an-overview}}

\textbf{Decision tree} is the most powerful and popular tool for
classification and prediction. A Decision tree is a flowchart like tree
structure, where each internal node denotes a test on an attribute, each
branch represents an outcome of the test, and each leaf node (terminal
node) holds a class label.

\hypertarget{decision-rules}{%
\subsection{Decision rules}\label{decision-rules}}

The decision tree can be linearized into decision rules, where the
outcome is the contents of the leaf node, and the conditions along the
path form a conjunction in the if clause. In general, the rules have the
form:

\begin{verbatim}
if condition1 and condition2 and condition3 then outcome.
\end{verbatim}

Decision rules can be generated by constructing association rules with
the target variable on the right. They can also denote temporal or
causal relations.

    \hypertarget{random-forest}{%
\section{Random Forest}\label{random-forest}}

The \emph{Random forest} is a classification algorithm consisting of
many uncorrelated decisions trees. It uses \textbf{bagging} and
\textbf{feature randomness} when building each individual tree to try to
create an uncorrelated forest of trees whose prediction by committee is
more accurate than that of any individual tree.

So how to ensure that the models are diversified ?

We can use the following two methods :

\begin{itemize}
\item
  \textbf{Bagging (Bootstrap Aggregation)} - Decision trees are very
  sensitive to the data they are trained on and small changes to the
  training set can result in significantly different tree structures.
  Random forest takes advantage of this by allowing each individual tree
  to randomly sample from the dataset with replacement, resulting in
  different trees. This process is known as bagging. With bagging we are
  not subsetting the training data into smaller chunks and training each
  tree on a different chunk. Rather, if we have a sample of size \(N\),
  we are still feeding each tree a training set of size \(N\) (unless
  specified otherwise). But instead of the original training data, we
  take a random sample of size \(N\) with replacement.
\item
  \textbf{Feature Randomness} - In a normal decision tree, when it is
  time to split a node, we consider every possible \emph{feature} and
  pick the one that produces the most separation between the
  observations in the \emph{left node} vs those in the \emph{right
  node}. In contrast, each tree in a random forest can pick only from a
  \emph{random subset of features}. This forces even more variation
  amongst the trees in the model and ultimately results in \emph{lower
  correlation} across trees and more \emph{diversification}.
\end{itemize}

    \begin{center}
    \adjustimage{max size={0.8\linewidth}{0.9\paperheight}}{d15.png}
    \end{center}
    { \hspace*{\fill} \\}

\begin{center}
    \textbf{Node splitting in a random forest model is based on a random
subset of features for each tree.}
\end{center}

The underlined feature is eventually selected, and the rest of the
diagram is self-explanatory.

    \hypertarget{comparison-to-logistic-regression}{%
\subsection{Comparison to Logistic Regression
:}\label{comparison-to-logistic-regression}}

None of them are ``better'' than the other, but following best practices
we can jot down a few points :

\begin{itemize}
\item
  If the problem/data is linearly separable, then first we should try
  logistic regression. If we don't know, then still we start with
  logistic regression because that will be our baseline, followed by
  non-linear classifier such as random forest.
\item
  If our data is categorical, then random forest should be our first
  choice; however, logistic regression can be dealt with categorical
  data.
\item
  If we want easy to understand results, logistic regression is a better
  choice because it leads to simple interpretation of the explanatory
  variables.
\item
  If speed is our criteria, then logistic regression should be our
  choice.
\item
  If the data is unbalanced, then random forest may be a better choice.
\item
  If number of data objects are less than the number of features,
  logistic regression should not be used.
\end{itemize}

    \hypertarget{conclusion}{%
\section{Conclusion}\label{conclusion}}

Now, that we have a rough idea about the working of quite a few
\textbf{regressors} and \textbf{classifiers}, we will move on into the
deep world of learning :)

\hypertarget{references}{%
\section{See Also}\label{references}}

\begin{itemize}
    \item 
        Hands-On Machine Learning with Scikit-Learn, Keras and TensorFlow by Aurélien Géron.
    \item
        Python Machine Learning by Sebastian Raschka and Vahid Mirjalili
    \item
        Deep Learning by Ian Goodfellow, Yoshua Bengio and Aaron Courville
    \item 
        \href{https://towardsdatascience.com/}{Towards Data Science}
\end{itemize}



    % Add a bibliography block to the postdoc
    
    
    
\end{document}
